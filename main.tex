%% RiSE Latex Template - version 0.5
%%
%% RiSE's latex template for thesis and dissertations
%% http://risetemplate.sourceforge.net
%%
%% (c) 2012 Yguaratã Cerqueira Cavalcanti (yguarata@gmail.com)
%%          Vinicius Cardoso Garcia (vinicius.garcia@gmail.com)
%%
%% This document was initially based on UFPEThesis template, from Paulo Gustavo
%% S. Fonseca.
%%
%% ACKNOWLEDGEMENTS
%%
%% We would like to thanks the RiSE's researchers community, the 
%% students from Federal University of Pernambuco, and other users that have
%% been contributing to this projects with comments and patches.
%%
%% GENERAL INSTRUCTIONS
%%
%% We strongly recommend you to compile your documents using pdflatex command.
%% It is also recommend use the texlipse plugin for Eclipse to edit your documents.
%%
%% Options for \documentclass command:
%%         * Idiom
%%           pt   - Portguese (default)
%%           en   - English
%%
%%         * Text type
%%           bsc  - B.Sc. Thesis
%%           msc  - M.Sc. Thesis (default)
%%           qual - PHD qualification (not tested yet)
%%           prop - PHD proposal (not tested yet)
%%           phd  - PHD thesis
%%
%%         * Media
%%           scr  - to eletronic version (PDF) / see the users guide
%%
%%         * Pagination
%%           oneside - unique face press
%%           twoside - two faces press
%%
%%		   * Line spacing
%%           singlespacing  - the same as using \linespread{1}
%%           onehalfspacing - the same as using \linespread{1.3}
%%           doublespacing  - the same as using \linespread{1.6}
%%
%% Reference commands. Use the following commands to make references in your
%% text:
%%          \figref  -- for Figure reference
%%          \tabref  -- for Table reference
%%          \eqnref  -- for equation reference
%%          \chapref -- for chapter reference
%%          \secref  -- for section reference
%%          \appref  -- for appendix reference
%%          \axiref  -- for axiom reference
%%          \conjref -- for conjecture reference
%%          \defref  -- for definition reference
%%          \lemref  -- for lemma reference
%%          \theoref -- for theorem reference
%%          \corref  -- for corollary reference
%%          \propref -- for proprosition reference
%%          \pgref   -- for page reference
%%4
%%          Example: See \chapref{chap:introduction}. It will produce 
%%                   'See Chapter 1', in case of English language.

\documentclass[pt,twoside,onehalfspacing,bsc]{risethesis}

\usepackage{natbib}
\usepackage{babel}
\usepackage{supertabular}
\usepackage{microtype}
\usepackage{lscape}
\usepackage{multirow}
\usepackage{tikz}
\usepackage{float}
\usepackage{url}
\usepackage{enumitem}

%% Aumenta a margem entre o label e as tabelas
% \usepackage{caption}
% \captionsetup[table]{skip=8pt}

%% Label at the bottom
\lstset{
	captionpos=b
}

%% Change the following pdf author attribute name to your name.
\usepackage[linkcolor=blue,citecolor=blue,urlcolor=blue,colorlinks,pdfpagelabels,pdftitle={Monografia de Victor Tavares},pdfauthor={Victor Tavares}]{hyperref}

\address{SALVADOR}

\universitypt{Universidade Federal da Bahia}

\departmentpt{Departamento de Ci\^{e}ncia da Computa\c{c}\~{a}o}

\programpt{}

\majorfieldpt{Ci\^{e}ncia da Computa\c{c}\~{a}o}
\majorfielden{Computer Science}

\title{SafeWatch: Sistema de Detecção de Quedas para Smartwatches}
\date{Outubro/2016}

\author{Victor de Souza Tavares}
\adviser{Vaninha Vieira}

\begin{document}

\frontmatter
\frontpage
\presentationpage

\begin{dedicatory}
Dedico esta dissertação à minha família, amigos e professores que me deram todo o apoio necessário para chegar até aqui.
\end{dedicatory}

\begin{epigraph}[]{John Woods}
Always code as if the guy who ends up maintaining your code will be a violent psychopath who knows where you live
\end{epigraph}

\resumo
% Escreva seu resumo no arquivo resumo.tex
Quedas podem ter sérias consequências, a ponto de serem consideradas um grave problema de saúde pública que afeta principalmente a população idosa, onde está relacionado com a perda de confiança, autoestima, e autonomia. Esse problema se mostra ainda mais relevante se consideramos o crescente número de idosos que em busca de sua independência e autonomia decidem morar sozinhos. É crucial que o idoso tenha rápido acesso ao atendimento médico, parte importante para a sua rápida recuperação. A demora no atendimento médico está ligado ao aumento de taxas de mortalidade e gravidade em um evento de queda. 

Pensando nisso, foi desenvolvido o \textit{SafeWatch}, um sistema de detecção de quedas embarcado em relógios inteligentes (em inglês, Smartwatch). O sistema proposto irá monitorar o idoso através de sensores presentes no smartwatch, e ao detectar uma queda, além de vibrar no pulso do usuário, irá informar para uma lista de contatos de emergência do usuário a sua localização e a possibilidade do idoso estar em uma situação de perigo. Experimentos foram realizados com oito indivíduos de biótipos distintos, onde cada um deles deveria simular uma queda em sentidos distintos. Através deste experimento, foi possível detectar o grau de confiabilidade da aplicação utilizando os valores de \textit{Sensibilidade} e \textit{Especificidade} que atingiram 89.06\% e 100\% respectivamente.




\begin{keywords}
	Relógios Inteligentes, Computação Ubíqua, Queda, Idoso
\end{keywords}

\abstract
% Write your abstract in a file called abstract.tex

Falling can have serious consequences, enough to be considered a serious public health problem that affects mainly the older population, in which is related to the loss of confidence, self-esteem and autonomy.This problem is shown even more relevant if we consider the growing number of seniors, who in the search of their independence and autonomy decide to live alone. It is crucial that the elderly have quick access to medical care, a key part for  quick recovery. The delay in medical care is linked with the increase of mortality rates and severity in a fall event. Thinking about it, the \textit{SafeWatch} was developed as a fall detection system embedded in smartwatches. The proposed system will monitor the seniors through sensors at the smartwatch, and when a fall is detected, It will vibrate on the user's wrist and report to a list of emergency contacts their location and inform the possibility of the elderly to be in a dangerous situation. Experiments were performed with eight individuals of different biotypes, in which each one of them simulated a fall event in different directions. According to the experiments, it was possible to evaluate the application reliability through the values of \textit{Sensitivity} and \textit{Specificity} that reached 89,06\% and 100\%, respectively.



\begin{keywords}
	Smartwatch, Ubiquitous Computing,  Fall, Elderly
\end{keywords}

% Summary (tables of contents)
\tableofcontents

% List of figures
\listoffigures

% List of tables
\listoftables

% List of acronyms
% Acronyms manual: http://linorg.usp.br/CTAN/macros/latex/contrib/acronym/acronym.pdf
\listofacronyms
\begin{acronym}[ACRONYM] 
% Change the word ACRONYM above to change the acronym column width.
% The column width is equals to the width of the word that you put.
% Read the manual about acronym package for more examples:
%   http://linorg.usp.br/CTAN/macros/latex/contrib/acronym/acronym.pdf

\acro{PNAD} { Pesquisa Nacional por Amostra de Domicílio }
\acro{SDQ} {Sistema de Detecção de Queda}
\acro{AD} {Atividades Diárias}
\acro{FOF} {Fear of Falling}
\acro{MEMS} { Sensores Microeletromecânicos }
\acro{SMV} {Soma da Magnitude Vetorial }
\acro{SMA} {Soma das Acelerações}
\acro{BLE} {Bluetooth Low Energy}
\acro{IDE} {Ambiente de Desenvolvimento Integrado}

    
\end{acronym}

% List of listings
\lstlistoflistings

\mainmatter
%\setcounter{page}{21}
\chapter{Introdução}
\label{cap:introducao}

\begin{quotation}[]{Mark Weiser}
First were mainframes, each shared by lots of people. Now we are in the personal computing era, person and machine staring uneasily at each other across the desktop. Next comes ubiquitous computing, or the age of calm technology, when technology recedes into the background of our lives. 






\end{quotation}

Devido a diversos avanços tecnológicos e médicos, a população mundial vem envelhecendo de forma gradual. Projeções feitas pelas \cite{unPopulation:13}, indicam que 11,57\% da população mundial tem 60 anos ou mais. Este mesmo relatório aponta que em 2050 a porcentagem de idosos irá quase dobrar, correspondendo a 21,1\% da população. Esta tendência não é muito diferente no Brasil, onde de acordo com as projeções do \cite{ibgePopulation:16}, 8,17\% da população irá ter 65 anos ou mais em 2016, com este número aumentando para 13,44\% em 2030.

Além de ser uma parcela da população que cresce, o número de idosos que moram sozinhos também vem aumentando. De acordo com o \cite{PNAD:12}, entre 1992 e 2012,  este número triplicou no Brasil, passando de 1,1 milhão para 3,7 milhões, um aumento de 215\%. Na busca pela sua independência, o idoso fica vulnerável a um dos principais problemas desta faixa etária, as quedas. De acordo com um estudo da Organização Mundial de saúde,  de 28\% a 35\% da população maior do que 64 anos sofrem pelo menos uma queda por ano. De acordo com o \cite{portalBrasilQuedas:12}, o SUS (Sistema Único de Saúde) registra a cada ano, um gasto de mais de R\$ 51 milhões com o tratamento de fraturas decorrentes de queda. Além de custosos, estas quedas representam um risco muito grande para o idoso, casos mais graves de fatura pode levar até morte, como por exemplo a fratura de fêmur com um índice de mortalidade de 30\%.

Outra questão que torna as quedas ainda mais prejudiciais a saúde física e mental do idoso é o tempo entre a queda e o  atendimento médico. De acordo com diversos estudos, a demora no atendimento está fortemente ligada ao índice de mortalidade e gravidade do acidente depois de uma queda. De acordo com X, quando ocorre o “long-lie”, ou seja, o idoso permanece mais de uma hora no chão a chance de que o idoso faleça antes dos 6 meses do ocorrido sobe para 50\%.

Visando minimizar essas graves consequências das quedas, diversos sistemas de detecção foram desenvolvidos nos últimos anos, porém estes sistemas não fazem uso de tecnologias mais popularizadas no mercado, ou utilizam de plataformas que não são vestíveis, prejudicando a mobilidade do usuário. 

Por exemplo, diversos sistemas utilizam o smartphone como principal plataforma na detecção de quedas. Analisando somente as questões de popularidade e hardware, o smartphone se apresenta uma solução plausível. De acordo com \cite{smartphoneSales:15}, foram vendidos  mais 1 bilhão de aparelhos somente em 2014. Na perspectiva de hardware, a maioria dos smartphones modernos possuem giroscópio ou  acelerômetro, dois dos principais sensores utilizados no reconhecimento de atividades atualmente.
 
Entretanto, quando pensamos em mobilidade, o smartphone passa a ser uma solução com baixo potencial,  pois para que os sistemas funcione corretamente o mesmo precisa está fixo em uma posição pré-estabelecida pelo sistema, como os bolso ou pulso do usuário \cite{FallDectionSmartPhone:12}. O que faz com que, em situações corriqueiras, como colocar o celular na bolsa, possa gerar um número grande de falsos positivos.

Este trabalho propõe como solução a criação de um sistemas de detecção de quedas através de uma solução integrada entre smartphone e o smartwatch. Na primeira vez que executar a aplicação, o usuário deverá cadastrar o nome e-mail dos usuários que ele deseja contactar em caso de uma  queda, feito este registro o usuário está pronto pra utilizar a aplicação. Ao detectar a queda, o smartwatch irá vibrar, e caso o usuário não indique que está bem, o sistema irá enviar um email com a localização do usuário para todos os contatos já cadastrados.   

O smartwatch é uma ferramenta que permite que este tipo de aplicação seja calma e invisível para o usuário, além de ter uma capacidade de processamento bastante similar aos smartphones com uma popularidade crescente. A Samsung, umas das empresas pioneiras no mercado de smartwatches, lançou em outubro de 2015 o Samsung Gear S2. O Gear é um exemplo de como esses sistemas estão cada vez mais poderosos. Ele possui uma memória RAM de 512 MB e 4GB de armazenamento,  conectividade WiFi e 4G além de diversos sensores como giroscópio e acelerômetro \cite{samsungSpecification:16}. A popularidade desta plataforma é vista através do número de smartwatches vendidos. No ano de 2015, 30,32 milhões de aparelhos foram vendidos, e a previsão é de que, em 2016, este número suba para 50,40 milhões. 


Os próximos capítulos estão organizados da seguinte maneira: O Capítulo \ref{cap:sistemasRecomendacao} apresenta os conceitos teóricos usados neste trabalho referente a Sistemas de Detecção de Quedas. O Capítulo \ref{cap:wearable_systems} se aprofunda nos sistemas de detecção de quedas que utilizam plataformas vestíveis; O Capítulo \ref{cap:safeWatch} apresenta o SafeWatch, o sistema de Detecção de Quedas desenvolvido através de uma solução integrada entre o smartphone e o smartwatch. O Capítulo \ref{cap:avaliacao} apresenta o experimento realizado, e realiza a avaliação da ferramenta. Por fim, no capítulo \ref{cap:conclusão}, seguem as conclusões e considerações finais. 


\chapter{Sistemas de Detecção de Quedas}
\label{cap:sistemasRecomendacao}

Um \ac{FDS}, pode ser descrito com um dispositivo de apoio, cujo principal objetivo é alertar o usuário em um evento de queda \citep{igual2013challenges}. Este tipo de sistema pode ser desenvolvido de diversas formas é pode está tanto embarcado em um dispositivo Wearable como um smartwatch ou se basear em um sistema de monitoramento utilizando câmeras.

Com o uso de sistemas de detecção de quedas é possível que o usuário tenha o seu medo de cair reduzido, e dependendo da solução que foi desenvolvida, possa ser socorrido de maneira muito mais rápida, caso necessário. Um individuo que que já sofreu uma queda, pode desenvolver uma sindrome chamada \ac{FOF}, que pode levar a perda da capacidade de se realizar atividades rotinieiras, como passear em um parque, ou assistir um filme em família \citep{legters2002fear}.

As seções desse capítulo são organizadas da seguinte maneira: A seção \ref{sec:fall_definition} define o conceito de quedas e expõe os diversos estados da mesma; A \ref{sec:fall_system_types} irá demonstrar os 3 diferentes tipos de sistemas de detecção quedas mais populares; A seção \ref{sec:sensors} irá demonstrar os principais sensores utilizados nos sistemas de detecção de quedas; A seção  Por fim, a seção \ref{sec:fds_examples}  irá mostrar exemplos de aplicações que realizam a detecção de quedas. 



\section{Definição de Queda}
\label{sec:fall_definition}
De forma geral, podemos definir uma queda como um evento súbito e involuntário, onde o indivíduo de uma posição em pé ou sentado, passa a ocupar uma posição integral ou parcialmente deitada (Horizontal). Na busca por uma definição mais formal, em 1987, o Kellog International Working Group on the prevention of falls, descreveu uma queda como "Vir ao chão ou algum nível mais baixo, sem a intenção como consequência de um golpe violento, perda de consciência, ou início súbito de paralisia como no caso de um acidente vascular cerebral ou um ataque epiléptico"  \citep{igual2013challenges}. Quando analisamos uma queda através das pesperctiva da aceleração do movimento, ela pode ser dividida em etapas descritas na imagem \ref{fig:fall_states}. Estas 4 etapas são as seguintes:



\begin{figure}[ht]
	\centering
	\includegraphics[scale=0.4]{imagens/fall_states.png}
	\caption{Etapas de uma queda \citep{hsieh2014wrist}.}
	\label{fig:fall_states}
\end{figure} 


\begin{itemize}
	\item{\textbf{Período Anterior a Queda (Pre-Fall)}: Durante este período o individuo estará realizando suas atividades cotidianas, que podem levar ou não a um pico de aceleração que deve ser tratado para que se possa evitar falso-positivos. Ações que geralmente levam a este pico de aceleração são movimentos como sentar ou se deitar muito rápido, ou dependendo da posicionamento dos sensores, atividades físicas que exigem bastante movimentação. }
	
	\item{\textbf{Período Anterior a Queda (Pre-Fall)}: Durante este período o individuo estará realizando suas atividades cotidianas, que podem levar ou não a um pico de aceleração que deve ser tratado para que se possa evitar falso-positivos. }
	
	\item{\textbf{Período Queda Livre (Free-Fall)}: Durante este período o individuo está se descolando em direção ao chão. Nesta fase o valor de sua aceleração irá tender a 0.  }
	
	\item{\textbf{Período do Impacto (Impact-Phase)}: Período caracterizado pelo impacto do índividuo, este período é crítico na aplicação, poís é onde ocorre o pico de aceleração. }
	
	\item{\textbf{Período de Inatividade (Inactive State)}: Período posterior a queda, onde o usuário irá realizar o esforço para se levantar. Em quedas mais graves, onde o usuário está incapaz de se movimentar ou inconsciente este valor se modificação de maneira muito sutil, porém de forma geral, o usuário que sofreu uma queda não se levanta imediatamente. De acordo com \cite{mehner2013location}, o período de pós impacto e inatividade levá aproximadamente 2 segundos.  }
	
\end{itemize}


\section{Tipos de Sistemas de Detecção de Queda}
\label{sec:fall_system_types}
section Tipos de sistema
\section{Sensores}
\label{sec:sensors}
sensores

\section{Exemplos de Sistemas de Detecção de Queda}
\label{sec:fds_examples}
examples


\section{Técnicas de Recomendação}
\label{sec:tecnicnasRecomendacao}

Para a realização da recomendação são utilizadas algumas técnicas de recomendação, feitas a partir da predição sobre as informações dos itens e usuários.

Várias técnicas de recomendação foram propostas como base para um sistema de recomendação, como as técnicas colaborativa, baseada em conteúdo, baseada em conhecimento e demográfica. As técnicas de recomendação podem ser diferenciadas com base em suas fontes de conhecimento: de onde vem o conhecimento necessário para fazer a recomendação? Em alguns sistemas, esse conhecimento é o conhecimento das preferências de outros usuários \citep{Burke:2007:HWR:1768197.1768211}.

A utilidade de um item para um usuário pode ser influenciada pelo conhecimento que o usuário tem do domínio, por exemplo, usuário iniciantes vs experientes de uma câmera digital, ou pode depender do momento em que a recomendação foi feita. Ou o usuário pode estar mais interessado em itens mais perto de seu local atual, um restaurante por exemplo. Assim, a recomendação deve ser adaptada para esses detalhes específicos adicionais \citep{ricci2011recommender}.

\cite{Burke:2007:HWR:1768197.1768211} apresentou quatro diferentes classes de técnicas de recomendação baseadas em fontes de conhecimento, como ilustrado na Figura \ref{fig:tec_recomendacao_fontes_conhecimento}:

\begin{itemize}
	\item{\textbf{Filtragem Colaborativa}: O sistema gera recomendações para um usuário utilizando apenas informações de itens de outros usuários com gosto similar.}
	
	\item{\textbf{Baseada em Conteúdo}: O sistema gera recomendação de duas fontes: as características associadas ao item e as classificações que os usuários deram a ele.}
	
	\item{\textbf{Demográfica}: Fornece recomendações baseado no perfil demográfico do usuário.}
	
	\item{\textbf{Baseada em Conhecimento}: Sugere itens baseado em inferências sobre as preferências e necessidades do usuário.}
\end{itemize}

\begin{figure}
	\centering
	\includegraphics[scale=0.8]{imagens/recommendation_techniques_knowledg_sources.png}
	\caption{Técnicas de recomendação e suas fontes de conhecimento \citep{Burke:2007:HWR:1768197.1768211}.}
	\label{fig:tec_recomendacao_fontes_conhecimento}
\end{figure} 

	A seguir é discutido em detalhes alguns dos tipos de recomendação.
	
\subsection{Filtragem Colaborativa}

Na filtragem colaborativa o sistema recomenda ao usuário ativo itens que usuários com gostos similares gostaram no passado. A semelhança de gostos de dois usuários é calculada baseada na similaridade do histórico de classificações dos usuários. A filtragem colaborativa é considerada a técnica em sistemas de recomendação mais popular e mais largamente implementada \citep{ricci2011recommender}.

A ideia chave é que a classificação de um usuário \textit{u} para um novo item \textit{i} é provável de ser similar para a de outro usuário \textit{v}, se \textit{u} e \textit{v} tem classificado outros itens de maneira similar. De modo similar, é provável que \textit{u} classifique dois itens \textit{i} e \textit{j} de forma semelhante, se outros usuários tem feito classificações similares para estes dois itens \citep{ricci2011recommender}.

Em \cite{Burke:2002:HRS:586321.586352} e \cite{Burke:2007:HWR:1768197.1768211}, tem-se que os métodos de filtragem colaborativa podem ser divididos nas duas classes gerais baseados em \textit{vizinhança} e \textit{modelo}. Na filtragem colaborativa baseada em vizinhança as classificações usuário-item armazenadas no sistema são utilizadas diretamente para predizer classificações para novos itens. Isto pode ser feito de duas maneiras conhecidas como recomendação \textit{user-based} ou \textit{item-based} \citep{ricci2011recommender}.

Sistemas baseados em usuário, avaliam o interesse de um usuário \textit{u} por um item \textit{i} utilizando as classificações, para este item, de outros usuários, chamados \textit{vizinhos}, que tem padrões de classificação similar. Os vizinhos do usuário \textit{u} são tipicamente os usuários \textit{v} cujas classificações para os itens classificados por \textit{u} e \textit{v}, isto é I{$_{uv}$}, são mais correlacionados com aqueles de \textit{u}. A abordagem baseada no item, por outro lado, prediz a classificação de um usuário \textit{u} para um item \textit{i} baseado nas classificações de \textit{u} para itens similares a \textit{i}. Em tais abordagens, dois itens são similares se muitos usuários do sistema tem classificado estes itens de maneira similar \citep{ricci2011recommender}.

Diferente dos sistemas baseado em vizinhança, que usam as classificações armazenadas diretamente na predição, abordagens baseadas em modelo usam essas classificações para aprender um modelo preditivo. A ideia geral é modelar as interações usuário-item com fatores representando características latentes dos usuários e itens no sistema, como a classe de preferência do usuário e a classe de categoria dos itens. Este modelo é então treinado utilizando os dados disponíveis, e depois utilizado para predizer classificações de usuários para novos itens \citep{ricci2011recommender}.

\subsection{Filtragem Baseada em Conteúdo}

Sistemas de recomendação baseado em conteúdo tentam recomendar itens similares a aqueles que um usuário gostou no passado, ao passo que sistemas que utilizam o paradigma de recomendação colaborativa identificam usuários que possuem preferências similares a um usuário e recomenda itens que eles tem gostado \citep{lops2011ContentBased}.

\begin{figure}
	\centering
	\includegraphics{imagens/content_based_architecture.png}
	\caption{Arquitetura de alto nível de um Sistema de Recomendação Baseado em Conteúdo \citep{lops2011ContentBased}.}
	\label{fig:content_based_architecture}
\end{figure} 

Sistemas implementando uma abordagem de recomendação baseada em conteúdo analisam um conjunto de documentos e/ou descrições de itens previamente classificados por um usuário, e constrói um modelo ou perfil de interesses do usuário baseado nas características dos objetos classificados por este usuário \citep{Mladenic:1999:TRI:630307.630472}. O perfil é uma representação estruturada dos interesses do usuário, adaptado para recomendar novos itens interessantes. O processo de recomendação consiste basicamente em combinar os atributos do perfil do usuário contra os atributos de um objeto de conteúdo. O resultado é um julgamento de relevância que representa os níveis de interesse do usuário naquele objeto. Se um perfil reflete com precisão as preferências do usuário, isso é uma grande vantagem para a eficácia no processo de acesso a informação. Por exemplo, poderia ser utilizado para filtrar os resultados de pesquisa para decidir se um usuário está interessado em uma específica página Web ou não e, em caso negativo, prevenir de ser exibida \citep{lops2011ContentBased}.

\subsection{Arquitetura de Sistemas de Recomendação Baseados em Conteúdo}

Segundo \cite{lops2011ContentBased}, em sistemas baseados em conteúdo é necessário utilizar técnicas apropriadas para representar os itens e produzir o perfil do usuário, e algumas estratégias para comparar o perfil do usuário com a representação do item. O processo de recomendação é realizado em três passos, cada qual é manipulado por um componente separado:

\begin{itemize}
	\item{\textbf{Analisador de Conteúdo}: Quando a informação não tem estrutura, como um texto, algum tipo de pré-processamento é preciso para extrair informação relevante e estruturada. A principal responsabilidade do componente é representar o conteúdo dos itens (por exemplo, documentos, páginas Web, notícias, descrição de produtos, etc.) vindos de fontes de informação de forma adequada para o próximo passo de processamento.}
	
	\item{\textbf{Aprendiz de Perfil}: Este módulo coleta dados representativos das preferências dos usuários e tenta generalizar estes dados, para construir o perfil do usuário. Normalmente, a estratégia de generalização é realizada através de técnicas de \textit{aprendizado de máquina}, que são capazes de inferir um modelo de interesses de usuário partindo de itens gostados ou não gostados no passado.}
	
	\item{\textbf{Componente de Filtragem}: Este módulo explora o perfil do usuário para sugerir itens relevantes através da combinação da representação do perfil do usuário contra os itens a serem recomendados. O resultado é um binário ou contínuo julgamento de relevância (computado utilizando alguma métrica de similaridade \citep{Herlocker:2004:ECF:963770.963772}), neste último caso resultando em uma lista ranqueada de itens potencialmente interessantes.}
\end{itemize}

A arquitetura de alto nível de um sistema de recomendação baseado em conteúdo está retratada na Figura \ref{fig:content_based_architecture}.


\subsection{Comparação das Técnicas de Recomendação}

Todas as técnicas de recomendação tem seus pontos fortes e fracos. \cite{Burke:2002:HRS:586321.586352} indicou alguns desses pontos, os quais são discutidos abaixo:

\begin{itemize}
	\item{\textbf{Usuário Novo}: Recomendações partem da comparação entre o usuário alvo e outros usuários baseado unicamente na acumulação de classificações, assim um usuário com poucas classificações é difícil de categorizar.}
	
	\item{\textbf{Item Novo}: Do mesmo modo, um item novo que não tem recebido muitas classificações também não pode ser facilmente recomendado: o problema do "item novo". É também conhecido como problema do "early rater", desde que a primeira pessoa a classificar um item recebe poucos benefícios por fazer isso. Isso torna necessário que sistemas de recomendação forneçam outros incentivos para encorajar usuários a fornecer classificações.}
\end{itemize}

Sistemas de recomendação colaborativos dependem das recomendações entre usuários e tem problemas quando o espaço de classificações é esparso, onde poucos usuários classificam os mesmos itens.

Estes três problemas sugerem que técnicas colaborativas puras são melhores para problemas onde a densidade de interesse do usuário é relativamente alta entre um pequeno e estático universo de itens. Se o conjunto de itens muda rapidamente, classificações antigas serão de pouco valor para novos usuários que não serão capazes de ter suas classificações comparadas com as dos usuários existentes. Se o conjunto de itens é grande e o interesse do usuário dissemina pouco, então a probabilidade de sobreposição com outros usuários será pequena \citep{Burke:2002:HRS:586321.586352}.

Sistemas de recomendação colaborativos funcionam melhor para um usuário que se encaixa em um nicho com muitos vizinhos de gosto semelhante. A técnica não funciona bem para o chamado "\textit{gray sheep}"~ \citep{claypool99}, que cai na fronteira entre panelinhas de usuários existentes. Isso também é um problema para sistemas demográficos que tentam categorizar usuários em características pessoais. Por outro lado, sistemas de recomendação demográficos não tem o problema do "usuário novo", porque eles não requerem um lista de classificações dos usuários. Ao invés, eles tem o problema de reunir as informações demográficas necessárias. \citep{Burke:2002:HRS:586321.586352}.

\section{Aplicações de Sistemas de Recomendação}

O primeiro sistema de recomendação foi um sistema experimental de filtragem de email, Tapestry \citep{goldberg1992tapestry}, desenvolvido por pesquisadores da Xerox Palo Alto Research Center \citep{Resnick:1997:RS:245108.245121}.

O objetivo do Tapestry era filtrar e arquivar os e-mails que chegavam todos os dias, de acordo com as opniões dadas pelas que efetuavam a leitura. Assim, este sistema utilizava uma abordagem colaborativa.

\subsection{TiVo}

Com o desenvolvimento das smart TVs, os usuários passaram a poder fazer classificações pela TV. TiVo\footnote{http://www.tivo.com} é um serviço de televisão muito utilizado nos Estados Unidos, que utiliza um sistema de recomendação para sugerir programas que sejam de interesse do usuário \citep{Ali:2004:TMS:1014052.1014097}. O TiVo permite que os usuários classifiquem os programas utilizando o controle remoto \ref{fig:tivo_recomendacao}.

O feedback implícito, por exemplo, se um filme está sendo gravado, é levado em consideração em adição da explícita classificação dos programas feita pelos usuários. Pedir ao usuários para responder perguntas é entediante e levanta questões de segurança, então o sistema tenta coletar as informações necessárias em background.

\begin{figure}
	\centering
	\includegraphics[scale=0.65]{imagens/TiVo_suggestion.png}
	\caption{Lista de Recomendação do Sistema de Recomendação do TiVo \citep{TiVoSuggestion}.}
	\label{fig:tivo_recomendacao}
\end{figure} 

O sistema do TiVo utiliza a Filtragem Colaborativa para fazer recomendações ao usuário com base nas preferências de usuários que tem gostos semelhantes. Também é utilizada a Filtragem Baseada em Conteúdo para fazer recomendações com base nas características (canal, título, gênero, atores) dos programas que o usuário já assistiu anteriormente.

\subsection{Biblioteca Digital}

Bibliotecas digitais são coleções de objetos digitais. Sistemas de Recomendação podem ser utilizados em uma aplicação de biblioteca digital para ajudar os usuários a localizarem e selecionarem informação e fontes de conhecimento \citep{Porcel:2010:DII:1663649.1663728}.

O CYCLADES\footnote{http://www.ercim.org/cyclades} \citep{rendaIPM05}, mostrado na Figura \ref{fig:cyclades}, é um ambiente colaborativo de arquivo virtual distribuído e aberto, que fornece diversos serviços para dar suporte a pesquisadores individuais como também a comunidade de pesquisadores, de uma maneira altamente personalizável. Os algoritmos de recomendação utilizam Filtragem Baseada em Conteúdo e Filtragem Colaborativa.


\begin{figure}
	\centering
	\includegraphics[scale=0.65]{imagens/cyclades.png}
	\caption{Página Web do CYCLADES. Figura elaborada pelo autor (2015).}
	\label{fig:cyclades}
\end{figure} 


\subsection{Amazon}

A Amazon\footnote{http://www.amazon.com} utiliza sistemas de recomendação para ajudar seus clientes a encontrar produtos para comprar \citep{Schafer:1999:RSE:336992.337035}.

\begin{figure}
	\centering
	\includegraphics[scale=0.65]{imagens/amazon.png}
	\caption{Recomendação de Livros no site da Amazon. Figura elaborada pelo autor (2015).}
	\label{fig:amazon}
\end{figure} 

\textbf{Customers Who Bought This Item Also Bought}: a Amazon é estruturada com uma página de informação para cada livro, dando detalhes do texto e informação de compra. O recurso \textit{Clientes que compraram} é encontrado na página de cada livro do catálogo, e recomenda livros frequentemente comprados por clientes que compraram o livro selecionado. Na Figura \ref{fig:amazon} pode ser visto um exemplo em que são recomendados livros que foram comprados por clientes que também compraram um determinado livro.

\textbf{Amazon Delivers}: Clientes selecionam em uma lista de categorias/gênero, e periodicamente recebem e-mails com as últimas recomendações nas categorias escolhidas.

\textbf{Book Matcher}: Este recurso permite que clientes façam avaliação direta sobre livros que eles leram, onde dão notas em uma escala de 0 a 5. Depois de avaliar uma amostra de livros, os clientes podem solicitar recomendações de livros que podem gostar. Nesse ponto, uma meia dúzia de livros não-avaliados são apresentados os quais se relacionam com o gosto do usuário.

\textbf{Comentários dos Clientes}: Este recurso permite que clientes recebam recomendações de livros baseado nas opiniões de outros clientes. Está localizado na página de informação de cada livro, e é uma lista de 1-5 estrelas de avaliação com comentários fornecidos por outros clientes que leram o livro em questão.

\section{Sumário}

Neste capítulo, vimos uma visão geral sobre os Sistemas de Recomendação. Começamos mostrando um histórico sobre Sistemas de Recomendação. Em seguida discutimos alguns conceitos sobre os dados utilizados em um sistema. Então, foi apresentado uma lista de tarefas desempenhadas por sistemas de recomendação. Discutimos também as principais técnicas de recomendação. Por fim, foi mostrado algumas aplicações de sistemas de recomendação. No capítulo \ref{cap:userModel} discutiremos sobre o Modelo de Usuário. Será introduzido do que se trata, formas e requisitos de modelagem, e os principais algoritmos utilizados.
\chapter{Sistemas Vestíveis} 
\label{cap:wearable_systems}

Sistemas Vestíveis podem ser definidos como dispositivos eletrônicos móveis que podem ser discretamente embutidos nos trajes do usuário, como parte da roupa ou um acessório. Diferente dos sistemas móveis convencionais, eles podem funcionar sem ou com muito pouca interferência nas atividades do usuário \citep{lukowicz2004wearable}. 

Hoje, muitos destes dispositivos vestíveis vem com uma gama de sensores embutidos que são utilizados na detecção de quedas. O tipo de sensor mais comum utilizado em \ac{SDQ} é o acelerômetro, com alguns desses sistemas também utilizando o giroscópio como um sensor auxiliar. De acordo com a revisão sistemática feito por \cite{igual2013challenges}, 186 dos 197 sistemas analisados utilizam o acelerômetro como sensor principal na detecção de quedas. A utilização desses sensores em \ac{SDQ} se deve muito pela popularização e o barateamento dos mesmos, além da utilização desses sensores embarcados em smartphones e smartwatches.


As seções desse capítulo são organizadas da seguinte maneira: A seção \ref{sec:sensors} descreve os principais tipos de sensores utilizados em \ac{SDQ}; A \ref{sec:sensor_position} fala sobre o posicionamento de sensores, fazendo uma correlação entre o posiciomanento dos sensores e a accurâcia dos sistemas estudados; A seção \ref{sec: FDS_algorithm} fala sobre os diferentes tipos de algoritmos de detecção de quedas mostrando suas vantagens e desvantagens; Por fim, a seção \ref{sec:FDS_examples} irá mostrar exemplos de aplicações que realizam a detecção de quedas atravês de tecnologias vestíveis. 



\section{Sensores}
\label{sec:sensors}
A escolha e o bom funcionamento de sensores são uma parte essencial em sistemas de detecção de quedas. De forma geral, sensores são dispositivos que convertem fenômenos físicos em sinais elétricos. Sendo assim , eles representam a camada de comunicação entre o mundo físico e o mundo digital. 

Os dois tipos principais de sensores utilizados em sistemas de detecção de queda são o acelerômetro e o giroscópio. Eles fazem parte de um grupo chamado de \ac{MEMS}. Estes sensores são geralmente feitos de chips de silício utilizando as mesmas técnicas usadas na confecção de chips de computadores pessoais. Para que um sensor possa ser classificado como \ac{MEMS} alguma parte do seu design precisa vibrar ou se mover de alguma forma \citep{milette2012professional}. 

De forma geral, tanto o acelerômetro quanto o giroscópio utilizam três eixos para expressar seus valores. O sistema de coordenadas utilizado é relativo a cada dispositivo. Na Figura \ref{fig:axis_device} temos um exemplo do sistema de coordenadas utilizado pelo sistema operacional Android\footnote{https://www.android.com/}. Neste sistema o ponto de origem é o centro da tela do dispositivo, quando segurado na posição vertical \citep{sensorAndroidDocs}. 

\begin{figure}[ht]
	\centering
	\includegraphics[scale=0.6]{imagens/axis_device.png}
	\caption{Sistema de coordenadas utilizado pelo sistema operacional Android \citep{sensorAndroidDocs}.}
	\label{fig:axis_device}
\end{figure} 

\subsection{Acelerômetro}
\label{subsec:accelerometer}
Fisicamente, o acelerômetro é um dispositivo composto de uma pequena massa  anexado a pequenas molas que são utilizadas para medir a aceleração aplicada sobre um dispositivo, incluindo a força da gravidade. A aceleração é medida analisando o quanto a massa se distancia do seu ponto de equilíbrio. 

\begin{figure}[ht]
	\centering
	\includegraphics[scale=0.6]{imagens/MEMS_device.png}
	\caption{Força sendo aplicando sobre uma massa presa a molas \citep{milette2012professional}.}
	\label{fig:acelerometro}
\end{figure} 

	
Na Figura \ref{fig:acelerometro} em A é possível ver um aparelho parado em uma mesa, sobre ele só irá agir a força de gravidade 1G de aproximadamente $9.8 m/s^{2}$. Em  B, o aparelho foi jogado para a direita, então irá agir sobre ele, além da força da gravidade, uma aceleração no sentido para onde o aparelho se movimentou. Já em C, vemos um aparelho em queda livre com aceleração no sentido oposto a força da gravidade, o que faz com a massa fique localizada em seu ponto de equilibrio, e a força resultante  seja de $0G$ \citep{milette2012professional}.


\subsection{Giroscópio}
	
	O giroscópio, similarmente aos acelerômetros, são pequenas massas em pequenas molas, só que em vez de medir a aceleração, são utilizados para medir um tipo de força chamada de Força de Coriolis. A Força de Coriolis é a tendência que um objeto livre possui de sair do curso quando visto de um ponto de referência em rotação \citep{milette2012professional}. Por exemplo, se sentarmos em um carrossel e rolarmos uma bola pra longe, a bola irá parecer desviar em uma linha reta, como se existisse uma força agindo sobre ela. Esta força é chamada de Força de Coriolis.
	
	Apesar de possuir uma estrutura física semelhante, o acelerómetro e o giroscópio se diferem em seu funcionamento. Em vez de esperar a força da gravidade agir sobre a massa, o giroscópio funciona vibrando esta massa sobre o eixo definido. Quando o giroscópio é rotacionado, a Força de Coriolis faz com que a massa comece a ser mover em um eixo diferente no qual ele estava vibrando anteriormente.  \cite{milette2012professional}.
	
	Como a Força de Coriolis age somente quando o dispositivo está em rotação, o giroscópio só é capaz de calcular a velocidade angular, ou seja, a velocidade com que o aparelho está sendo rotacionado. 
	
	A orientação do dispositivo irá definir quando o valor da rotação será positivo ou negativo. Nos dispositivos Android, a velocidade Angular é medida em radianos por segundo ($rad/s$) e é positiva em rotações no sentido anti-horário \citep{GyroscopeAndroidDocs}. 
	
\section{Posicionamento de Sensores}
\label{sec:sensor_position}

O posicionamento dos sensores afeta diretamente a performance dos \ac{SDQ}, dependendo da posição onde colocamos os sensores, o sistema pode indicar uma maior ou menor quantidade de falhas. 

Não existe um consenso sobre a posição otimizada dos sensores para que se possa realizar a detecção de quedas. De acordo com \cite{abbate2011recognition}, a cintura seria o local ideal para o posicionamento, já que estaria mais perto do centro de gravidade do corpo humano. Entretanto em \cite{kangas2007determination}, já foi sugerido que a cabeça seria o melhor lugar para posicionar os sensores. Outras soluções, como em \cite{gjoreski2011accelerometer}, já propõem o uso de mais de um sensor, colocando-os em diferentes partes do corpo com o objetivo de aumentar ainda mais a precisão dos \ac{SDQ}.


De acordo com \cite{casilari2015analysis}, o pulso não é o local recomendado para o posicionamento de sensores em sistemas de detecção de quedas. Isso se deve a constante movimentação dos braços, que podem gerar um número grande de falso-positivos. Entretanto, alguns sistemas tem conseguido resultados satisfatórios com \ac{SDQ} localizados no pulso. O sistema proposto por \cite{hsieh2014wrist}, foi capaz detectar quedas em 151 das 160 quedas simuladas e obteve uma especificidade (capacidade de não reconhecer eventos de quedas, como tal) de 95\%.

Além da performance do sistema, outras questões precisam ser levadas em consideração quando pensamos no posicionamento dos sensores. Uma delas é a danificação dos sensores na ocorrência de uma queda. Caso o sistema pare de funcionar, um possível alerta de emergência poderá não ser enviado e o idoso poderá estar correndo grande perigo. 

Outra questão é a usabilidade, o uso de muitos sensores, apesar de poder elevar a precisão do sistema, poderá levar a um desconforto do usuário, que pode fazer até com que o mesmo desista de usá-lo. 


\section{Algoritmos de Detecção de Queda}
\label{sec: FDS_algorithm}
Os algoritmos de detecção de quedas recebem como entrada os dados obtidos através dos sensores e são capazes de determinar se o que ocorreu foi um evento de queda ou somente uma \ac{AD}. De acordo com \cite{casilari2015analysis}, é possível separar os algoritmos de detecção em dois grandes grupos: Algoritmos de detecção através de métodos de reconhecimento de padrões e algoritmos baseados em limiares. 


\subsection{Reconhecimento de Padrões}
O reconhecimento de padrões é uma área do aprendizado de máquina que foca no reconhecimento de padrões e regularidades de dados \citep{anzai2012pattern}. Diversas técnicas de aprendizado de máquina tem sido empregadas na detecção de quedas, de acordo com a revisão sistemática feita por \cite{casilari2015analysis}, algoritmos como o de \textit{Naïve Bayes}, \textit{Redes Neurais}, e \textit{Árvores de Decisão} tem sido utilizados.


Um exemplo de sistema que utiliza a técnica de reconhecimento de padrões foi proposto por \cite{zhao2012fallalarm}. O seu algoritmo de detecção de quedas analisa dados do acelerômetro através de uma árvore de decisão, assim identificando um evento de queda. 

Este algoritmo é composto de 2 fases, a primeira é o que chamamos em aprendizado de máquina de fase de treinamento. Será realizado a coleta de dado e a  extração das características que são pertinentes, além do treinamento de um modelo de árvore de decisão. Na figura \ref{fig:decision_tree} podemos ver o modelo de árvore de decisão gerado.  Este modelo de árvore de decisão é capaz de reconhecer atividades como andar, correr, estado estático ou um evento de queda através das variáveis \textit{Std\_x}, \textit{Mean\_y} e \textit{Slope} que representam caracteristicas do sistema. 


\begin{figure}[ht]
	\centering
	\includegraphics[scale=0.6]{imagens/decision_tree.png}
	\caption{ Modelo de árvore de decisão construida em \cite{zhao2012fallalarm}.}
	\label{fig:decision_tree}
\end{figure} 


Na segunda fase, chamada de fase de testes, o modelo de árvore de decisão é utilizado em uma aplicação no smartphone que será responsável pelo reconhecimento de atividades. Os dados dos sensores são catalogados e transformados nas características do sistema, que serviram como dados de entrada da árvore de decisão, que terá como saída o tipo de atividade que foi desempenhada.

Sistemas que utilizam reconhecimento de padrões, normalmente está suscetível a altos custos computacionais, análise massiva de dados, e acesso a grandes bancos de dados ou longos períodos de treinamentos onde o algoritmo de classificação precisa ser parametrizado e adaptado a diferentes grupos de usuários \citep{casilari2015analysis}. Em contraste, existem os algoritmos baseados em limiares que tendem a ser mais simples e similarmente eficientes, desde que encontremos limiares adequados.

\subsection{Baseado em Limiares}
Algoritmos baseados em limiares utilizados na detecção de quedas comparam os dados dos sensores com um ou mais valores pré-definidos, chamados de limiares. Estes valores podem ser fixos ou adaptados. Quando estes valores são adaptados, eles não mudam dinamicamente enquanto os usuários estão utilizando o sistema. Em vez disso, o usuário irá introduzir dados sobre o seu perfil fisiológico e o sistema irá informar os limiares adequados \citep{habib2014smartphone}. Um exemplo deste tipo de sistema pode ser visto em \cite{sposaro2009ifall}, o valor limiar mudar de acordo com os parâmetros providos pelo usuário como altura, peso e nível de atividade.

De acordo com a revisão sistemática feita por \cite{casilari2015analysis}, muitos sistemas de detecção de queda utilizam o valor de \ac{SMV} do vetor de aceleração como  valor limiar principal em seus algoritmos de detecção. De acordo com \cite{casilari2015analysis}, o valor de \ac{SMV} é definido através da equação em \ref{eq:SMV}, onde $X_i$, $Y_i$, $Z_i$, representam, respectivamente, os valores de aceleração dos eixos x, y, z obtidos através do acelerômetro.

\begin{equation}
SMV = \sqrt{X_i^2 + Y_i^2 + Zi_i^2} 
\label{eq:SMV}
\end{equation}

A escolha dos limiares é um fator determinante para o sucesso deste tipo de algoritmo de detecção de quedas. A escolha dos limiares pode ser feita através de experimentos preliminares como em \cite{zhang2013honey}. Em seu trabalho, um grupo de voluntários foi escolhido para realizar diversas \ac{AD}, como andar, correr, subir e descer escadas e também realizar a simulação de quedas. Através dos dados obtidos foi possível descobrir os limiares de \ac{SMV} para um evento de queda.  

De acordo com \cite{cao2012falld}, a performance dos algoritmos aumenta significativamente quando utilizamos valores de limiar dinâmicos. De acordo com sua pesquisa, o número de eventos de queda que não foram caracterizados como tal, caíram de 53 para 29 quando o peso, sexo idade foram levados em consideração no momento da definição dos limiares.

Outra questão importante quando utilizado este tipo de algoritmo, é o número de limiares utilizados. De acordo com \cite{casilari2015analysis}, o uso de um único limiar faz com que o algoritmo emita um número grande de alertas falsos, categorizando \ac{AD} como eventos de queda, fazendo com que o uso de somente um limiar não seja adequado no desenvolvimento de \ac{SDQ}.


\section{Trabalhos Relacionados}
\label{sec:FDS_examples}
Como vimos no decorrer desse trabalho, não existe na literatura um algoritmo ou dispositivo padrão para o desenvolvimento de \ac{SDQ}. A plataforma vestível é bastante promissora por estar naturalmente aclopada a alguma parte do corpo do usuário, sendo possivel criar um algoritmo de detecção mais confiáveis, já que, como visto em \cite{casilari2015analysis}, grande parte dos algoritmos de detecção de quedas tem como pré-condição para o seu bom funcionamento, que o dispositivo esteja localizado em uma posição fixa do corpo, como cintura, pulso ou cabeça. Sendo assim, falaremos sobre três sistemas de detecção de quedas que embarcaram os seus sistemas de detecção de quedas em plataformas vestíveis. Estes três sistemas foram escolhidos pela semelhança dos mesmos com a solução proposta neste trabalho.

\subsection{SPEEDY - Detector de Quedas em um Relógio de Pulso}
SPEEDY foi a primeiro protótipo de um relógio detector de quedas construído em um smartwatch \citep{degen2003speedy}. Em seu trabalho, ele  utilizou dois sensores que são capazes de medir a aceleração através de 3 eixos \textit{x, y, z}. Na Figura \ref{fig:speedy} é possível ver o protótipo do sistema desenvolvido e os 3 eixos utilizados para calcular o valor da aceleração. 

\begin{figure}[ht]
	\centering
	\includegraphics[scale=0.5]{imagens/speedy.png}
	\caption{ SPEEDY e seus eixos \citep{degen2003speedy}.}
	\label{fig:speedy}
\end{figure} 

O algoritmo de detecção de quedas do Speedy utiliza um algoritmo baseado em limiares com 3 valores distintos: o valor de \ac{SMV} calculado através da formula que pode ser vista em \ref{eq:SMV}, dois valores de velocidade distintos chamados de $v_1$ e $v_2$. A velocidade $v_1$ é o valor aproximado da velocidade vertical (de queda), e o valor da velocidade $v_2$ representa a velocidade do dispositivo Speedy. 

No primeiro passo do algoritmo, um alto valor de velocidade precisa ser identificado indicando uma possível queda. Depois disso, nos próximos 3 segundos um impacto precisa ser detectado, representado por alto valores de aceleração. Depois disso, o usuário é observado por mais 60 segundos, se durante este tempo, pelo menos 40 segundos forem marcados por inatividade um alerta sonoro é emitido. 

O Speedy foi avaliado através de quedas simuladas por 3 indivíduos em um colchão. Cada indivíduo simulou  quedas em 3 posições diferentes: Frente, lado e costas. Foram realizadas um total de 45 quedas, onde 65\% delas foram corretamente marcadas como um evento de queda. 

O Speedy foi desenvolvido utilizando um dispositivo próprio, fazendo com que os usuários precisem adquirir um dispositivo vestível que irá realizar exclusivamente a detecção de quedas fazendo com que a sua adesão seja mais complexa do que em um sistema que já utilize smartwatches já consolidados no mercado.


\subsection{F2D - Sistema de Detecção de Quedas}
\label{subsec:F2D_System}
F2D é uma applicação Android embarcada em um smartwatch \textit{AW-420.RX} da Simvalley Mobile\footnote{http://www.simvalley-mobile.de/} \citep{kostopoulos2015f2d}. O algoritmo implementado no F2D tem como entrada os dados do acelerômetro, levando em consideração os movimentos realizados depois de um evento de queda e a localização do usuário. 

Para que se possa detectar as quedas, o F2D utiliza um algoritmo baseado em limiares, onde os limiares foram definidos utilizando um banco de dados com mais de 150 eventos simulados de queda. O algoritmo de detecção de quedas presente no F2D é composto de quatro etapas \citep{kostopoulos2015f2d}:

\begin{enumerate}
	
	\item{\textbf{Padrão de Queda}: Para que um evento possa ser identificado como uma possível queda, o valor da aceleração precisa ultrapassar um limite que varia de $10 m/s^{2}$ a $18 m/s^{2}$ que representa o impacto da queda, e depois de um intervalo de tempo, precisa ultrapassar um limiar de $2 m/s^{2}$  a $7 m/s^{2}$, caracterizado como o movimento residual da queda. Tanto os valores exatos dos limiares, quanto o intervalo de tempo entre a análise dos mesmos variam de acordo com o  perfil do usuário.   }
	
	\item{\textbf{Módulo de Decisão}: Toda vez que ambas as condições da etapa anterior são satisfeitas é acrescido 1 em um contador. São estabelecidos dois valores $X$ e $Y$. O valor de $X$ representa o limiar do contador e $Y$ representa o seu limite. Sendo assim, o valor do contador precisa ficar entre X e Y, ou seja, $ X \leq Contador < Y $. Caso este valor seja maior que Y, outra atividade estava sendo desempenhada, como por exemplo correr, e caso este valor do contador seja menor que X, onde $X = 1$, então um movimento brusco do braço aconteceu, mas que não caracteriza uma queda.  }
	
	\item{\textbf{Ação posterior à Queda}: Logo depois que um evento é caracterizado como queda, o F2D é capaz de identificar se o usuário conseguiu se recuperar, e voltou a exercer suas atividades normais, caso isto ocorra, o sistema não irá emitir um alerta para o seu cuidador, caso contrário, um alerta é emitido }
	
	\item{\textbf{Ação baseada na Localização}: O F2D também se baseia na localização do usuário. Caso este esteja na rua, e todas as etapas anteriores se concretizaram, um alarme é enviado para o cuidador. Caso o usuário esteja em casa, o sistema faz uso da tecnologia \textit{iBeacon} para categorizar certos locais como seguros ou potencialmente perigosos. O iBeacon utiliza o sistema de detecção de proximidade do \ac{BLE} para enviar um identificador único para aplicações ou sistemas operacionais compatíveis que estejam ao alcance do mesmo \citep{kostopoulos2015f2d}.  Caso o local esteja marcado como potencialmente perigoso, uma mensagem é enviada ao cuidador, caso contrário, o usuário terá a oportunidade de cancelar o envio, em um possível evento de queda.}     	
\end{enumerate}

O \textit{F2D}, além de realizar a detecção de quedas, utiliza dados do contexto para definir se o evento ocorrido foi potencialmente perigoso ou não, o que o diferencia da maioria das soluções presentes na literatura. Entretanto, para que ele possa mapear essas áreas é necessário a montagem de toda uma infraestrutura com a utilização de iBeacons, o que torna a sua implementação e adesão mais complexa.


\subsection{Sistema de Detecção de Quedas de Pulso}
O sistema proposto por \cite{hsieh2014wrist} utiliza dois dispositivos vestíveis acoplados no pulso do usuário como pode ser visto na Figura \ref{fig:wrist_worn}. 


\begin{figure}[ht]
	\centering
	\includegraphics[scale=0.4]{imagens/wrist_worn.png}
	\caption{ Dispositivos vestíveis circulados. \citep{hsieh2014wrist}.}
	\label{fig:wrist_worn}
\end{figure} 


Cada um dos dispositivos está equipado com um módulo Zigbee\footnote{http://www.zigbee.org/} responsável pela transmissão dos dados e  um acelerômetro e  giroscópio de 3 eixos. A frequência tanto do acelerômetro quanto do giroscópio foram configuradas para $50 Hz$, ou seja, os dados são coletados a cada \textit{20 ms}.


O algoritmo proposto utiliza ambos os dados do acelerômetro e do giroscópio para realizar a detecção de quedas. Os dados do giróscopio funcionam como um filtro inicial, desconsiderando a maioria das ativididades cotidianas, enquanto os dados do acelerômetro são responsáveis por realizar o julgamento final. O algoritmo utilizado, assim como os demais, é baseado em limiares onde os limiares foram definidos atravês de um treinamento inicial. 

O algoritmo proposto é capaz de diferenciar \ac{AD} como bater palmas e deitar-se, de um evento de queda em 95\% dos casos. A principal desvantagem do sistema proposto é a necessidade de se utilizar 2 dispositivos, podendo torna-se desconfortável para o usuário.

O sistema de detecção de pulso, assim como o \textit{Speedy}, utiliza de um dispositivo próprio para realizar a detecção, o que faz com sua adesão seja mais complexa. Além disso, para realizar a detecção o sistema necessita que o usuário utilize 2 relógios, o que pode ser incômodo para o usuário final. Por fim, ele necessita dos dados do giroscópio além do acelerômetro, o que pode exigir um maior custo computacional e consequentemente um maior consumo de bateria. 









\chapter{SafeWatch: Sistema de Detecção de Quedas}
\label{cap:safeWatch}

Um evento de queda pode ser bastante prejudicial a saúde do indíviduo, pricipalmente de um idoso. Pensando nisso, diversos tipos sistemas de detecção de quedas foram desenvolvidos. Neste trabalho, apresentamos o SafeWatch, uma solução integrada entre smartphone e smartwatch, onde quedas são detectadas de maneira automatizada, e se necessário, os contatos de emergência do idoso são informados de sua localização para que se possa prestar socorro de forma mais rápida possível. 


As seções desse capítulo são organizadas da seguinte maneira: A seção \ref{sec:architecture} mostra a arquitetura que foi definida e utilizada pela ferramenta construída; A Seção \ref{sec:tools} mostra as ferramentas que foram utilizadas para auxiliar a construção do SafeWatch; A Seção \ref{sec:implementation} descreve detalhes da implementação do SafeWatch; Por fim, a seção \ref{sec:screens} ilustra a execução da ferramenta.



\section{Arquitetura}
\label{sec:architecture}

De acordo com \cite{garlan1993introduction}, a arquitetura de um software define o sistema em termos de componentes e as interações existentes entre esses componentes. Em outras palavras, a arquitetura de software tem o objetivo de mostrar uma visão completa do sistema. 

O SafeWatch foi desenvolvido para funcionar como um aplicativo Android Wear\footnote{https://www.android.com/wear/} para smartwatches que funciona em conjunto com o smartphone Android do usuário, atravês de uma aplicação homônima, que está sincronizado com o mesmo. A ferramenta foi desenvolvida com base em uma arquitetura pré-definida e possui os seus módulos desacoplados para facilitar futuras mudanças ou melhorias. A ferramenta está dividida em 5 módulos como ilustra a figura \ref{fig:architecture}.

\begin{figure}[ht]
	\centering
	\includegraphics[scale=0.65]{imagens/architecture.png}
	\caption{ Arquitetura do SafeWatch. Figura Elaborada pelo autor (2016).}
	\label{fig:architecture}
\end{figure} 



\begin{itemize}
	\item{\textbf{Sensor Reader (Parte 1 da figura \ref{fig:architecture})}: Responsável pela configuração e gerenciamento dos sensores, mais especificamente do único sensor utilizado, o acelerômetro. }
	
	\item{\textbf{Fall Detector (Parte 2 da figura \ref{fig:architecture})}: Recebe informações provindas do \textit{Sensor Reader} para através do algoritmo de detecção de quedas categorizar um determinado evento como queda ou não.}.
	
	\item{\textbf{Watch Communicator (Parte 3 da figura \ref{fig:architecture})}: Realizará a comunicação entre o smartwatch e o smartphone do usuário. Irá receber dados do smartwatch que são tratados pela módulo chamado de \textit{Fall Handler}.}
	
	\item{\textbf{Fall Handler (Parte 4 da figura \ref{fig:architecture})}: Será responsável pelas ações do smartphone após um evento de queda, como o envio de emails e o gerenciamento dos dados do acelerômetro recebidos do smartwatch.}
	
	\item{\textbf{Contact Manager (Parte 5 da figura \ref{fig:architecture})}: Será responsável pelas gereciamento dos contatos de emergência do usuário. Ações como visualização, adição e remoção de contatos estão encapsuladas neste módulo.}
			
\end{itemize}

Os módulos \textit{Sensor Reader} e \textit{Fall Detector} estão presentes na aplicação embarcada no smartwatch, os demais módulos estão presentes na aplicação para smartphones. 

\section{Ferramentas Utilizadas}
\label{sec:tools}
Durante o desenvolvimento do \textit{SafeWatch} foram utilizadas diversas ferramentas que serviram para dar suporte a sua implementação e execução. Tanto o aplicativo para smartphones, quanto o aplicativo embarcado no smartwatch foram desenvoldidos utilizando o Android Studio\footnote{https://developer.android.com/studio/}. O Android Studio é a \ac{IDE} oficial do Google no desenvolvimento de aplicações móveis ou vestíveis. 


Nas classes do projeto relacionadas as telas do aplicativo foi utilizado o ButterKnife\footnote{texthttps://github.com/JakeWharton/butterknife}. O Butterknife é responsável por fazer a ligação entre os arquivos responsáveis pela criação das telas, e as classes que utilizam os componentes visuais destas telas. 

Para realizar os cálculos do desvio padrão foi utilizada a biblioteca do Apache chamada Commons-Math\footnote{http://commons.apache.org/proper/commons-math/}. Este biblioteca contém um conjunto de funções matemáticas e de estatística não presentes na biblioteca padrão do JAVA. 

Para que possamos enviar os dados do acelerômetro do smartwatch para o smartphone, eles precisam estar codificados em algum padrão, o padrão escolhido foi o JSON. O JSON é um formato de dados utilizado para comunicação entre dispositivos \citep{JSON;16}. Para que possamos fazer a codificação e decodificação dos dados do acelerômetro foi utilizado a biblioteca chamada Gson\footnote{http://www.json.org/}.

Por fim, para o envio de emails para os contatos de emergência foi utilizado as biblioteca padrão criada pela Oracle\footnote{http://www.oracle.com/} chamada de JavaMail\footnote{http://www.oracle.com/technetwork/java/javamail/index.html}. 





\section{Implementação}
\label{sec:implementation}
A implementação do SafeWatch foi dividida em várias partes, onde cada um delas é representada por um módulo independente dos demais. A linguagem de programação utilizada foi Java\footnote{http://www.oracle.com/technetwork/java/index.html}, linguagem padrão no desenvolvimento de aplicações Android. Nas seções abaixo são detalhados detalhes da implementação e funcionamento de cada módulo.


\subsection{Sensor Reader}
O módulo \textit{Sensor Reader} é responsável pela configuração e gerenciamento do acelerômetro. Aqui, o acelerômetro é configurado para atualizar seus dados a um frequência de 50 $Hz$, ou seja, a cada 20 ms. Os dados do acelerômetro são coletados a todo momento, mesmo quando a aplicação não está em primeiro plano.

Este módulo também será responsável por armazenar os dados do acelerômetro nos últimos 0.4 segundos para posterior uso no algoritmo de detecção, caso necessário. Tanto a escolha da frequência de 50 $Hz$ quanto o tempo de 0.4 segundos para o armazenamento de dados do acelerômetro serão explicados com mais detalhes em \ref{subsec:fall_detector}.



\subsection{Fall Detector}
\label{subsec:fall_detector}
O módulo \textit{Fall Detector} encapsula o algoritmo de detecção de quedas baseado em limiares utilizado pelo SafeWatch. Este é o módulo mais complexo da aplicação, pois nele se encontra a lógica responsável por decidir, através dos dados obtidos do acelerômetro, se um evento de queda ocorreu ou não. O algoritmo proposto é uma adaptação do algoritmo desenvolvido por \cite{hsieh2014wrist}. O algoritmo proposto neste trabalho se diferencia do algoritmo proposto em \cite{hsieh2014wrist} pela não utilização do giroscópio como sensor auxiliar, acredita-se que sem o uso do giroscópio é possível obter-se resultados satisfatórios como visto em \ref{subsec:F2D_System}. 

Um grande desafio quando utilizamos um algoritmo baseado em limiares é a definição dos valores dos limiares. Caso este valor seja muito alto, o sistema irá deixar escapar alguns eventos de queda, mas não irá categorizar uma \ac{ADL} como uma queda.  Do outro lado, se este valor for muito baixo, o sistema irá detectar todos os eventos de queda, mas algumas \ac{ADL} pode ser categorizadas como eventos de queda de maneira equivocada. De acordo com o treinamento inicial realizado por \cite{hsieh2014wrist}, o valor de \ac{SMV}, representado pela fórmula \ref{eq:SMV}, será maior que $6G$, onde $G \approx 9.8 m/s^2$,  no momento do impacto em um evento de queda. 

Também foi identificado por \cite{hsieh2014wrist}, que caso o valor de aceleração atinja o valor de $6G$, o valor do desvio padrão ficava com valores em torno de $1.07G$  em movimento regulares do braço realizados 0.4 segundos antes ou depois deste pico de aceleração.Entretanto, em eventos de queda este valor estava mais próximo de $1.69G$. 

Por fim, o periodo de inatividade posterior a uma queda foi analisado. De acordo com \cite{hsieh2014wrist}, o valor da \ac{SMA}, expresso atráves da equação \ref{eq:SMA}, tem uma relação diretamente proporcional com o nível de movimentação de um corpo. Foi identificado que em eventos de queda, o indíviduo tende a ficar parado por pelo menos 2 segundos, com valores de SMA inferiores a $200G$. E importante resaltar que este valor de $200G$ é encontrado quando a frequência do acelerômetro é de 50 $Hz$.Caso contrário, o número de amostras coletados será diferente, afetando diretamente no valor de $SMA$. 

\begin{equation}
SMA = \sum_{i=1}^{N} (\mid X_i\mid + \mid Y_i \mid + \mid Zi_i \mid)
\label{eq:SMA}
\end{equation}

Nesta equação $X_i$, $Y_i$, $Z_i$, são os valores da aceleração no tempo $i$ e $N$ é o número de amostras desejadas. Levando como base o treinamento inicial descrito acima foi possível desenvolver o algoritmo descrito na imagem \ref{fig:flow_chart}.

\begin{figure}[ht]
	\centering
	\includegraphics[scale=0.75]{imagens/flowChartAlgorithm.png}
	\caption{ Fluxograma do algoritmo proposto. Figura Elaborada pelo autor (2016).}
	\label{fig:flow_chart}
\end{figure} 

	\begin{enumerate}
		\item Os valores do acelerômetro são monitorados, caso $SMV$ seja maior do que $6G$, os demais valores de $SMV$ são monitorados por mais 0.4 segundos. O maior dos valores observado neste tempo é marcado como pico de aceleração e o algoritmo prossegue para o passo 2.
		\item O desvio padrão de \textit{SMV} é calculado, nos 0.4 segundos anteriores e posteriores a detecção do maior valor de SMV. Caso este valor não seja menor do que $1.5G$,  o algoritmo irá para o passo 3.
		\item O valor de $SMA$ é calculado, e caso este valor seja menor do que $ 200G $ finalmente confirmamos que um evento de queda ocorreu.
	\end{enumerate}

 Caso um evento de queda seja detectado, o relógio irá vibrar por 15 segundos, na espera de um feedback do usuário informando se ele está bem ou não. 

\subsection{Watch Communicator}
O módulo \textit{Watch Communicator} é responsável pela comunicação entre o smartwatch é o smartphone do usuário. Fisicamente, está comunicação é realizada via bluetooth, já a nível de software está comunicação é realizada através do que chamamos na arquiterura Android de \textit{Services}.

No Android, um \textit{Service} é um componente da aplicação capaz de realizar operações de longa duração em background e não provém uma interface com o usuário \cite{servicesAndroidDocs}. No SafeWatch os \textit{Services} recebem os dados do acelerômetro referentes ao evento de queda, ou seja todos os registros do acelerômetro 0.4 segundos antes do pico de aceleração, até 2 segundos depois deste valor. Além disso, também é enviado, uma variável booleana indicando se devemos ou não enviar um e-mail para a lista de contatos de emergência do usuário. Os emails de emergência são enviados para a lista de contato do usuário 15 segundos após um evento de queda, ou antes disso, caso o usuário confirme que precisa de ajuda. 

Para que os contatos da lista de emergência não sejam incomodados desnecessariamente na ocorrência de falsas detecções de quedas, o usuário poderá cancelar o envio dos emails de emergência, dentro de 15 segundos após um evento de queda, caso ele informe que está bem.

\subsection{Fall Handler} 
Este módulo é responsável pelo envio de e-mails e a manipulação de arquivos com os dados de uma queda. Para que se possa realizar o envio de e-mails, foi criado um email padrão do SafeWatch através do Gmail\footnote{https://mail.google.com}. O envio de e-mail é feito de forma assincrona, sem bloquear a interação do usuário com a aplicação. 

Os dados referentes a um evento de queda são salvos na raiz do sistema de arquivos do smartphone android na pasta \textit{/SafeWatch/smartwatch}. O arquivo é nomeado com o padrão \textit{experimentData\_timeStamp}, onde t\textit{timeStamp} representa o momento do salvamento do arquivo, em milisegundos. O arquivo está salvo no formato \textit{CSV}, com os valores de aceleração nos eixos x,y,z, o valor de \ac{SMV} correspondente ao registro e o tempo em milisegundos em que ele ocorreu. Apesar destes valores não serem de grande uso para o usuário final, ele poderão servir para o aprimoramento do aplicativo através da análise dados para verificação de limiares. 

\subsection{Contact Manager}
Neste módulo estão encapsulados as ações de adição, remoção e listagem dos contatos de emergência do usuário. Além do e-mail, nas informações do contato também constará o nome completo do mesmo. 


\section{Telas e Funcionamento}
\label{sec:screens}
O SafeWatch foi desenvolvido para funcionar como uma aplicativo android, atravês de uma solução integrada entre smartphone e smartwatch. O sistema foi desenvolvido de forma que o usuário necessite interagir o mínimo possível com os aplicativos tanto no smartphone quanto no smartwatch. 

De forma geral, a aplicação smartwatch irá monitorar as atividades do usuário através do acelerômetro e no momento que uma queda for detectada emitirá um alerta vibratório juntamente com um sinal para o smartphone. No smartphone está presente uma aplicação de gerenciamento geral do sistema, nesta aplicação o usuário será capaz de adicionar, visualizar e remover os contatos de emergência que seriam notificados no momento de uma queda.

Inicialmente, o usuário deve realizar o cadastro dos contatos de emergência, a tela de cadastro pode ser vista na figura \ref{fig:add_contact}. O usuário necessita informar o nome completo e email do contato desejado. Todos os contatos de emergência do usuário podem ser visualizados em forma de lista como pode ser visto na figura \ref{fig:list_contacts}.


Depois de adicionar os contatos de emergência, o usuário não necessita realizar mais nenhum tipo de cadastro ou configuração no sistema. Na ocorrência de uma queda o sistema irá se comportar como pode ser visto na figura \ref{fig:diagram}.


\begin{figure}[ht]
	\centering
	\includegraphics[scale=0.6]{imagens/DiagramaQueda.png}
	\caption{Aplicação no evento de queda. Figura Elaborada pelo autor (2016).}
	\label{fig:diagram}
\end{figure}

Enquanto um evento de queda não é detectado, o smartwatch irá realizar o monitoramento constante dos dados do acelerômetro, durante esta fase, o sistema irá mostrar uma mensagem, indicando que está realizando o monitoramento, como pode ser visto na figura \ref{fig:monitor_screen}. 


Na figura \ref{fig:monitor_fall}, é possível ver o momento que um evento de queda é detectado. Neste momento, o smartwatch irá emitir um sinal de vibração, além de uma mensagem perguntando se está tudo bem com o usuário. Este mensagem ficará visível por 15 segundos, caso o usuário não cancele o envio, uma mensagem é enviada para o smartphone informando que uma queda ocorreu.


O smartphone irá enviar um email para o usuário, o modelo do email pode ser visto na figura \ref{fig:mail_template}. Além de uma mensagem informando que o usuário pode está em uma situação de perigo, um link do \textit{Google Maps}\footnote{https://www.google.com/maps} é anexado com a última localização do usuário obtida pelo sistema. 


\begin{figure}[ht]
	\centering
	\includegraphics[scale=0.3]{imagens/tela_adicionar_contatos.png}
	\caption{Tela de adição de um contato de emergência. Figura Elaborada pelo autor (2016).}
	\label{fig:add_contact}
\end{figure} 


\begin{figure}[ht]
	\centering
	\includegraphics[scale=0.3]{imagens/tela_contatos.png}
	\caption{Tela de visualização dos contatos de emergência. Figura Elaborada pelo autor (2016).}
	\label{fig:list_contacts}
\end{figure}


\begin{figure}[ht]
	\centering
	\includegraphics[scale=0.12]{imagens/screen_fall.png}
	\caption{ Tela apresentada quando uma queda é detectada. Figura Elaborada pelo autor (2016).}
	\label{fig:monitor_fall}
\end{figure}



\begin{figure}
	\centering
	\includegraphics[scale=0.1]{imagens/screen_monitor.png}
	\caption{ Tela de monitoramento presente no smartwatch. Figura Elaborada pelo autor (2016).}
	\label{fig:monitor_screen}
\end{figure}


\begin{figure}[ht]
	\centering
	\includegraphics[scale=0.7]{imagens/mail_example.png}
	\caption{ Modelo de email enviado no evento de queda. Figura Elaborada pelo autor (2016).}
	\label{fig:mail_template}
\end{figure}


\chapter{Estudo Experimental}
\label{cap:avaliacao}

Neste capítulo será apresentado o processo de avaliação utilizado para verificar a precisão do sistema de detecção de quedas proposto. Espera-se que o SafeWatch apresente uma precisão similar aos demais sistemas de detecção de quedas presentes na literatura.

As seções desse capítulo são organizadas da seguinte maneira: A Seção \ref{sec:metodology} apresenta os detalhes da metodologia utilizada para avaliar o SafeWatch; A Seção \ref{sec:metrics} mostra as métricas utilizadas na avaliação; A Seção \ref{sec:results} apresenta os resultados obtidos no experimento realizado e faz uma comparação com os resultados obtidos em outros trabalhos.



\section{Metodologia}
\label{sec:metodology}

Para avaliarmos o desempenho do SafeWatch foi realizado uma série de experimentos. O algoritmo de detecção de quedas proposto foi avaliado através de um conjunto de quedas simuladas e também um conjunto de atividades diárias realizadas pelos participantes do experimento. O grupo de voluntários possui um perfil diversificado, sendo composto de 3 homens e 5 mulheres como pode ser visto na Tabela \ref{tab:experiment}. O experimento não foi realizado com nenhum idoso devido a grande dificuldade de simular eventos de queda sem por em risco a integridade física do mesmo. Além disso, as base de dados encontradas referentes a eventos de quedas com idosos são privadas e não foram disponibilizadas, como vista no trabalho proposto por \cite{kostopoulos2015f2d}.


\begin{table}
	\centering
	\caption{Participantes do experimento}
	\label{tab:experiment}
	\begin{tabular}{c|c|c|c|c}
		\hline
		\textbf{Indivíduo}  & \textbf{Idade} 	& \textbf{Sexo}   &    \textbf{Peso}    & \textbf{Altura} 	 \\
		Indivíduo 1         &    28          & Masculino            & 82kg      		& 1.80m          \\  
		Indivíduo 2         &    20          & Feminino             & 63kg      		& 1.63m          \\
		Indivíduo 3         &    25          & Feminino             & 62kg      		& 1.59m          \\ 
		Indivíduo 4         &    29          & Masculino            & 130kg      		& 1.65m          \\ 
		Indivíduo 5         &    27          & Feminino             & 57kg      		& 1.69m          \\ 
		Indivíduo 6         &    14          & Feminino             & 45kg      		& 1.62m          \\
		Indivíduo 7         &    20          & Masculino            & 88kg      		& 1.80m          \\ 
		Indivíduo 8         &    30          & Feminino             & 62kg      		& 1.55m          \\   
	\end{tabular}
\end{table}

 
O smartwatch escolhido para realizar o experimento foi um \textit{Moto 360} da 1º geração com as seguintes características \citep{moto360}:

	\begin{enumerate}
		\item Sistema Operacional: Android Wear 2.0.
		\item CPU: Qualcomm Snapdragon 400, 1.2 GHz.
		\item Memória RAM: 512 MB.
		\item Capacidade de Armazenamento: 4 GB.
	\end{enumerate}
	
Já o smartphone escolhido foi um \textit{LG G2} com as seguintes características \citep{lg_g2}:

	\begin{enumerate}
		\item Sistema Operacional: Android  Lollipop 5.0.2.
		\item CPU: Quad-core 2.26 GHz Krait 400.
		\item Memória RAM: 2 GB.
		\item Capacidade de Armazenamento: 16GB.
	\end{enumerate}


Para avaliar o algoritmo de detecção de quedas implementado, foi realizado o seguinte experimento, composto de três etapas:

\begin{itemize}
	\item{\textbf{Preparação}: Nesta etapa é solicitado que o usuário coloque o smartwatch em seu pulso e ajuste a pulseira do relógio de uma maneira que o smartwatch permaneça firme, mas confortável. Depois disso, é informado que o usuário deverá simular duas quedas em cada um dos sentidos escolhidos: Costas, Frontal, Lado Direito, Lado Esquerdo. A ordem das quedas é decidida pelo usuário, o único requisito é que ele realize todas as oito quedas. }
	
	\item{\textbf{Realização das Quedas}: O usuário irá se posicionar de pé, na frente de um colchão coberto de almofadas como pode ser visto na Figura \ref{fig:fall_image}. A partir desta posição, ele irá realizar as oitos quedas, duas de cada tipo, como descrito na etapa anterior. A ordem das quedas é escolhida pelo usuário.  }
	
	\item{\textbf{Realização de Atividades Diárias}: Nesta etapa solicitamos que o usuário realize 4 atividades do seu cotidiano. Elas são: sentar em uma cadeira, levantar de uma cadeira, deitar no colchão e levantar do colchão. Estas atividades são realizadas afim de verificar se o sistema proposto é capaz de distinguir atividades diárias de um evento de queda. O sistema proposto não é capaz de distinguir qual a atividade diária que está sendo realizada, ele só é capaz de diferencia-la de um evento de queda.}
	
	
\end{itemize}


Além de executarem as simulações de queda antes das atividades diárias, ambos os eventos foram executados de maneira isolada, ou seja, não foi seguida nenhuma ordem ou sequência pré-definida de eventos. 

\begin{figure}[ht]
	\centering
	\includegraphics[scale=0.25]{imagens/fall_image.png}
	\caption{Usuário em preparação para uma queda de costas. Figura Elaborada pelo autor (2016).}
	\label{fig:fall_image}
\end{figure} 


\section{Métricas de Avaliação}
\label{sec:metrics}

Para que possamos analisar a performance do sistema de detecção de quedas foram utilizadas três métricas, a \textit{Sensibilidade},  \textit{Especificidade} e \textit{Acurácia}. De acordo com \cite{casilari2015automatic}, os valores de \textit{Sensibilidade} e \textit{Especificidade} são duas métricas bastante utilizadas na literatura para a análise de performance em sistemas de detecção de quedas. Elas representam, respectivamente, a proporção de eventos de queda e \ac{AD} que foram classificadas corretamente como tal. Já a \textit{Acurácia} é uma combinação da \textit{Sensibilidade} e da \textit{Especificidade} e nos dá uma ideia geral da performance do sistema.

A \textit{Sensibilidade} é expressa pela fórmula \ref{eq:Sensibilidade}. As variáveis \textit{TP} e \textit{FN} são, respectivamente, acrónimos para True Positive (Verdadeiro Positivo em inglês) e False Negative (Falso Negativo em inglês). A variável TP representa o número de eventos de queda corretamente classificadas, enquanto FN representa as quedas que não foram detectadas pelo sistema.

\begin{equation}
Sensibilidade = \frac{TP}{TP + FN}
\label{eq:Sensibilidade}
\end{equation}


Já a \textit{Especificidade} é expressa pela fórmula \ref{eq:Especificidade}. As variáveis \textit{TN} e \textit{FP} são, respectivamente acrónimos para True Negative (Verdadeiro Negativo em inglês) e False Positive (Falso Positivo em inglês). A variável TN representa o número de atividades diárias corretamente classificadas como tal, enquanto FP representa as atividades diárias que foram classificadas como queda.

\begin{equation}
Especificidade = \frac{TN}{FP + TN}
\label{eq:Especificidade}
\end{equation}

Para calcularmos a acurácia geral do sistema, devemos utilizar a fórmula \ref{eq:accuracy}. 

\begin{equation}
Acuracia = \frac{TP + TN}{TP + FP + TN + FN}
\label{eq:accuracy}
\end{equation}


\section{Resultados}
\label{sec:results}

Na tabela \ref{tab:results_fall} podemos ver os resultados dos experimentos de queda para cada um dos indivíduos. Como descrito na Seção \ref{sec:metodology} cada individuo realizou duas quedas em quatro sentidos diferente. Cada elemento da tabela representa o número de quedas que foram identificadas com sucesso pelo sistema em cada uma das direções. 

\begin{table}[h]
	\centering
	\caption{Resultados do experimentos de Queda.}
	\label{tab:results_fall}
	\begin{tabular}{c|c|c|c|c}
		\hline
		\textbf{Individuo}  & \textbf{Frente} 	& \textbf{Costas}   &    \textbf{Direita}    & \textbf{Esquerda} 	 \\
		Individuo 1         & 2        		    & 1            		& 2      		 		 & 2         \\  
		Individuo 2         & 1        		    & 2            		& 2      		 		 & 2         \\
		Individuo 3         & 2        		    & 2            		& 1      		 		 & 2         \\
		Individuo 4         & 2        		    & 2            		& 2      		 		 & 2         \\
		Individuo 5         & 2        		    & 1            		& 2      		 		 & 1         \\
		Individuo 6         & 2        		    & 2            		& 2      		 		 & 1         \\
		Individuo 7         & 2        		    & 2            		& 2      		 		 & 2         \\
		Individuo 8         & 2        		    & 1            		& 2      		 		 & 2         \\
	\end{tabular}
\end{table}


Como podemos ver na tabela \ref{tab:results_fall}, o número de verdadeiros-positivos é de 57, enquanto o número de falsos-negativos é de somente 7, o que nos dá uma \textit{especificidade} de $89,06\%$. Foi possível observar que o limiar inicial de $6G$ no algoritmo de detecção de quedas não foi alcançado em todos os eventos erroneamente não categorizados como queda.


Já na tabela \ref{tab:results_adl}, podemos ver os resultados do experimentos de \ac{AD} para cada um dos indivíduos. Cada um deles realizou quatro tipos de \ac{AD}, como descrito na Seção \ref{sec:metodology}. Cada elemento da tabela representa o número de \ac{AD} que não foram identificadas pelo sistema como um evento de queda. 


\begin{table}[h]
	\centering
	\caption{Resultados do experimentos de Atividades Diárias.}
	\label{tab:results_adl}
	\begin{tabular}{c|c|c|c|c}
		\hline
		\textbf{Individuo}  & \textbf{Levantar(Cadeira)} 	& \textbf{Sentar(Cadeira)}   &    \textbf{Levantar(Cama)}    & \textbf{Deitar(Cama)} 	 \\
		Individuo 1         & 2        		    & 2            		& 2      		 		 & 2         \\  
		Individuo 2         & 2        		    & 2            		& 2      		 		 & 2         \\
		Individuo 3         & 2        		    & 2            		& 2      		 		 & 2         \\
		Individuo 4         & 2        		    & 2            		& 2      		 		 & 2         \\
		Individuo 5         & 2        		    & 2            		& 2      		 		 & 2         \\
		Individuo 6         & 2        		    & 2            		& 2      		 		 & 2         \\
		Individuo 7         & 2        		    & 2            		& 2      		 		 & 2         \\
		Individuo 8         & 2        		    & 2            		& 2      		 		 & 2         \\
	\end{tabular}
\end{table}

 

Neste experimento, nenhum dos 64 eventos \ac{AD} foi identificado como um evento de queda pelo sistema, levando a uma \textit{especificidade} de $100\%$. De forma geral o sistema possui uma \textit{acurácia} de $94,53\%$.


Como visto na Tabela \ref{tab:compare}, o algoritmo proposto neste trabalho apresentou resultados satisfatórios identificando um evento de queda em quase $90\%$ dos casos, e não apresentando nenhum falso-positivo nas \ac{AD} testadas. Em comparação com o Speedy, sistema desenvolvido por \cite{degen2003speedy}, o nosso sistema apresentou uma sensibilidade $24,06\%$ maior, ou seja, o SafeWatch apresentou uma especificidade de $89,06\%$ contra $65\%$ do Speedy em experimentos bastantes similares.

\begin{table}[h]
	\centering
	\caption{Comparação com os resultados dos Trabalhos Relacionados.}
	\label{tab:compare}
	\begin{tabular}{c|c|c|c|c}
		\hline
		\textbf{}  & \textbf{SafeWatch} 	& \textbf{Speedy}   &    \textbf{F2D}    & \textbf{Sistema de Detecção de Pulso} 	 \\
		Sensibilidade         & 89,06\%        		    & 65\%           & 93,48\%      		 		 & 95\%         \\  
		Especificidade        & 100\%        		    & -           	 & 98,54\%      		 		 & 96,07\%         \\
		Acurácia        	  & 94,53\%        		    & -              & 96,01\%      		 		 & 95,85\%         \\

	\end{tabular}
\end{table}

Entretanto em comparação com \cite{hsieh2014wrist}, este trabalho apresentou uma \textit{sensibilidade} $5,94\%$ menor e uma \textit{especificidade} $3,3\%$ maior. Um dos prováveis motivos desse valor menor de \textit{Sensibilidade} são as condições do experimento.  Os experimentos realizados por \cite{hsieh2014wrist} foram em uma superfície acolchoada mais fina que um colchão convencional, o que leva a uma aceleração maior no impacto, podendo diminuir o número de falsos-negativos no experimento realizado. Em relação ao tipos de quedas realizados, tanto este, quanto o trabalho proposto por \cite{hsieh2014wrist} apresentaram quatro tipos de quedas: frontais, laterais (Direito e Esquerdo), Costas. Já em relação as \ac{AD}, ele apresentou um número maior de atividades, incluindo andar e correr, o que pode levar a número maior de falso-positivos. 







\chapter{Conclusão}
\label{cap:conclusão}

A proposta deste trabalho foi criar um sistema de detecção de quedas que exige-se o mínimo de iteração possível do usuário, consumindo o mínimo de recursos possíveis com uma precisão similar aos demais \ac{FDS} existentes através de uma plataforma vestível que esteja se popularizando no mercado. 

As maiores dificuldades encontradas foram no desenvolvimento do algoritmo de detecção de quedas. Com a proposta inicial de somente o acelerômetro, diferente de demais sistemas que também utilizam o giroscópio a acurácia do sistema poderia ser afetada, caso algum dos limiares não se adaptassem a essa nova proposta. Outra dificuldade foi entender as caracteristicas de uma queda a partir de sua aceleração. Alguns conceitos físicos não são tão triviais, o que levou a muita pesquisa.  

Por fim, foi desenvolvido o SafeWatch, um sistema de detecção embarcado em um smartwatch que utiliza o smartphone como uma plataforma auxiliar. A interface é simples, e permite que o usuário interaja com o sistema de maneira fácil e somente quando necessário. O SafeWatch foi analisado em questão de perfomance e apresentou resultados satisfatórios com uma acurácia similar ou melhor que outros \ac{FDS} existentes.

Como trabalhos futuros, existem as seguintes possibilidades:

	\begin{enumerate}
		\item Buscar meios de otimizar o consumo de bateria do smartwatch mesmo com o constante monitoramento dos dados dos sensores.
		\item Integração com outros tipos de sensores, como o sensor de batimento cardiaco, afim de aumentar ainda mais a precisão do sistema.
		\item Reconhecimento de outros tipos de atividades além da queda (e.g detector de possível AVC),  visando expandir o sistema de um simples sistema de detector de quedas, para um sistema completo de monitoramento. 
		\item Realizar o teste da aplicação com um número maior de \ac{ADL}, principalmente aquelas que exigem uma movimentação maior do braço do individuo, como andar ou correr. 
 
	\end{enumerate}






\bibliographystyle{natbib}
\addcontentsline{toc}{chapter}{\bibliographytocname}
\bibliography{references}

% Appendix
%\clearpage
%\addappheadtotoc
%\appendix
%\appendixpage
%\chapter{Formulário}
\label{ap:formulario}


\def \tick{
$[$\hspace{0.3cm}$]$
}

\def \twooption#1#2{
\tick #1.  \tick #2.
}

\def \threeoption#1#2#3{
\tick #1.\newline
\tick #2.\newline
\tick #3.
}

\def \fouroption#1#2#3#4{
\tick #1.\newline
\tick #2.\newline
\tick #3.\newline
\tick #4.
}

\def \fiveoption#1#2#3#4#5{
\tick #1.\newline
\tick #2.\newline
\tick #3.\newline
\tick #4.\newline
\tick #5.
}
\def \datefield{
%date fied used in time sheets
    /\hspace{0.4cm}/
}

\def \rcolor{
%table row color
    \rowcolor[gray]{0.9}
}

\def \hcolor{
    \rowcolor[gray]{0.7}
}


\section{Formulário para o usuário do NLP Discovery}
\label{ap:sec:feedback}

\textbf{Nome: }
\\
\line(1,0){380}
\\
\textbf{Você tem algum conhecimento técnico a respeito de Serviços Web? }
\\
\twooption{Sim}{Não}
\\
\textbf{Encontrou alguma dificuldade ao utilizar a ferramenta?}
\\
\twooption{Sim}{Não}
\\
\textbf{Caso tenha respondido "Sim", detalhe a dificuldade encontrada:}
\\
\line(1,0){410}
\\
\line(1,0){410}
\\
\line(1,0){410}
\\
\line(1,0){410}
\\
\textbf{Qual foi a consulta realizada na ferramenta (Digite exatamente o mesmo texto que digitou na caixa de pesquisa da ferramenta)?  }
\\
\line(1,0){410}
\\
\line(1,0){410}
\\
\line(1,0){410}
\\
\line(1,0){410}
\\
\textbf{A busca gerou algum resultado relevante?}
\\
\twooption{Sim}{Não}
\\
\textbf{Caso tenha respondido "Sim", qual a posição do Serviço relevante no resultado?}
\\
\line(1,0){100}
\\


\end{document}