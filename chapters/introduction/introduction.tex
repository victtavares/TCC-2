\chapter{Introdução}
\label{cap:introducao}

\begin{quotation}[]{Mark Weiser}
First were mainframes, each shared by lots of people. Now we are in the personal computing era, person and machine staring uneasily at each other across the desktop. Next comes ubiquitous computing, or the age of calm technology, when technology recedes into the background of our lives. 






\end{quotation}

Devido a diversos avanços tecnológicos e médicos, a população mundial vem envelhecendo de forma gradual. Projeções feitas pelas \cite{unPopulation:13}, indicam que 11,57\% da população mundial tem 60 anos ou mais. Este mesmo relatório aponta que em 2050 a porcentagem de idosos irá quase dobrar, correspondendo a 21,1\% da população. Esta tendência não é muito diferente no Brasil, onde de acordo com as projeções do \cite{ibgePopulation:16}, 8,17\% da população irá ter 65 anos ou mais em 2016, com este número aumentando para 13,44\% em 2030.

Além de ser uma parcela da população que cresce, o número de idosos que moram sozinhos também vem aumentando. De acordo com o \cite{PNAD:12}, entre 1992 e 2012,  este número triplicou no Brasil, passando de 1,1 milhão para 3,7 milhões, um aumento de 215\%. Na busca pela sua independência, o idoso fica vulnerável a um dos principais problemas desta faixa etária, as quedas. De acordo com um estudo da Organização Mundial de Saúde,  de 28\% a 35\% da população maior do que 64 anos sofrem pelo menos uma queda por ano. De acordo com o \cite{portalBrasilQuedas:12}, o SUS (Sistema Único de Saúde) registra a cada ano, um gasto de mais de R\$ 51 milhões com o tratamento de fraturas decorrentes de queda.

Outra questão que torna as quedas ainda mais prejudiciais à saúde física e mental do idoso é o tempo entre a queda e o  atendimento médico. De acordo com \cite{bookFallOlderPeople:01}, a demora no atendimento está fortemente ligada ao índice de mortalidade e gravidade do acidente depois de uma queda. Na ocorrência do “long-lie”, ou seja, o idoso permanece mais de uma hora no chão, a chance de que o idoso faleça antes dos 6 meses do ocorrido sobe para 50\% \citep{wild1981dangerous}.

Visando minimizar essas graves consequências das quedas, diversos sistemas de detecção foram desenvolvidos nos últimos anos, porém estes sistemas não fazem uso de tecnologias mais popularizadas no mercado, ou utilizam plataformas que não são vestíveis, prejudicando a mobilidade do usuário. Por exemplo, diversos sistemas utilizam o smartphone como principal plataforma na detecção de quedas. Analisando somente as questões de popularidade e hardware, o smartphone se apresenta como uma solução plausível. De acordo com \cite{smartphoneSales:15}, foram vendidos mais de 1 bilhão de aparelhos somente em 2014. Na perspectiva de hardware, a maioria dos smartphones modernos possuem giroscópio ou  acelerômetro, dois dos principais sensores utilizados no reconhecimento de atividades atualmente.
 
Entretanto, quando pensamos em mobilidade, o smartphone passa a ser uma solução com baixo potencial,  pois para que os sistemas funcionem corretamente, os mesmos precisam estar fixos em uma posição pré-estabelecida, como o bolso ou pulso do usuário \citep{FallDectionSmartPhone:12}. Isso faz com que, em situações corriqueiras, como colocar o celular na bolsa, possa gerar um número grande de falsos positivos.

Este trabalho propõe como solução a criação de um sistema de detecção de quedas através de uma solução integrada entre smartphone e o smartwatch. O smartwatch será responsável pela detecção dos eventos de queda, enquanto o smartphone será responsável pelo gerenciamento dos contatos e envio das mensagens de emergência. 

O smartwatch é uma ferramenta que permite que este tipo de aplicação seja calma e invisível para o usuário, além de ter uma capacidade de processamento bastante similar aos smartphones, com uma popularidade crescente. A Samsung, umas das empresas pioneiras no mercado de smartwatches, lançou em outubro de 2015 o Samsung Gear S2. O Gear é um exemplo de como esses sistemas estão cada vez mais poderosos. Ele possui uma memória RAM de 512 MB e 4GB de armazenamento,  conectividade WiFi e 4G além de diversos sensores como giroscópio e acelerômetro \cite{samsungSpecification:16}. A popularidade desta plataforma é vista através do número de smartwatches vendidos. No ano de 2015, 30,32 milhões de aparelhos foram vendidos, e a previsão é de que, em 2016, este número suba para 50,40 milhões. 

Este trabalho ter por objetivo desenvolver um sistema de detecção de quedas utilizando os smartwatches já popularizados no mercado como ferramenta principal no desenvolvimento desta solução. O sistema deverá exigir o mínimo de iteração possível do usuário, sendo capaz de detectar as quedas de maneira automática. Também é desejável que o sistema utilize o mínimo de sensores possível, visando diminuir o custo computacional da solução apresentada, mantendo-se, a acurácia semelhante aos principais sistemas de detecção presentes na literatura.

Para realizar este trabalho a seguinte metodologia foi adotada: inicialmente investigamos os sistemas de detecção de quedas existentes na literatura, como foco nos algoritmos de detecção utilizados. Em seguida desenvolvemos o \textit{SafeWatch}, utilizando um algoritmo de detecção de quedas que utiliza somente os dados do acelerômetro como entrada. Por fim, elaboramos um experimento para avaliar a solução proposta.  

Participaram do nosso experimento oito pessoas, sendo três homens e cinco mulheres. Simulamos quatro tipos de queda e quatro tipos de atividades diárias. Os resultados indicaram que o sistema possui uma acurácia de 94,17\%.
 

Os próximos capítulos estão organizados da seguinte maneira: O Capítulo \ref{cap:sistemasRecomendacao} apresenta os conceitos teóricos usados neste trabalho referente a Sistemas de Detecção de Quedas. O Capítulo \ref{cap:wearable_systems} se aprofunda nos sistemas de detecção de quedas que utilizam plataformas vestíveis; O Capítulo \ref{cap:safeWatch} apresenta o SafeWatch, o sistema de Detecção de Quedas desenvolvido através de uma solução integrada entre o smartphone e o smartwatch. O Capítulo \ref{cap:avaliacao} apresenta o experimento realizado, e realiza a avaliação da ferramenta. Por fim, no capítulo \ref{cap:conclusão}, seguem as conclusões e considerações finais. 

