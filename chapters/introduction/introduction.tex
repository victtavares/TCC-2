\chapter{Introdução}
\label{cap:introducao}

\begin{quotation}[]{Steve Jobs}
You can’t connect the dots looking forward; you can only connect them looking backwards. So you have to trust that the dots will somehow connect in your future. You have to trust in something – your gut, destiny, life, karma, whatever. This approach has never let me down, and it has made all the difference in my life.
\end{quotation}

Devido a diversos avanços tecnológicos e médicos, a população mundial vem envelhecendo de forma gradual. Projeções feitas pelas \cite{unPopulation:13}, indicam que 11,57\% da população mundial tem 60 anos ou mais. Este mesmo relatório aponta que em 2050 a porcentagem de idosos irá quase dobrar, correspondendo a 21,1\% da população. Esta tendência não é muito diferente no Brasil, onde de acordo com as projeções do \cite{ibgePopulation:16}, 8,17\% da população irá ter 65 anos ou mais em 2016, com este número aumentando para 13,44\% em 2030.

Além de ser uma parcela da população que cresce, o número de idosos que moram sozinhos também vem aumentando. De acordo com o \cite{PNAD:12}, entre 1992 e 2012,  este número triplicou no Brasil, passando de 1,1 milhão para 3,7 milhões, um aumento de 215\%. Na busca pela sua independência, o idoso fica vulnerável a um dos principais problemas desta faixa etária, as quedas. De acordo com um estudo da Organização Mundial de saúde,  de 28\% a 35\% da população maior do que 64 anos sofrem pelo menos uma queda por ano. De acordo com o \cite{portalBrasilQuedas:12}, o SUS (Sistema Único de Saúde) registra a cada ano, um gasto de mais de R\$ 51 milhões com o tratamento de fraturas decorrentes de queda. Além de custosos, estas quedas representam um risco muito grande para o idoso, casos mais graves de fatura pode levar até morte, como por exemplo a fratura de fêmur com um índice de mortalidade de 30\%.

Outra questão que torna as quedas ainda mais prejudiciais a saúde física e mental do idoso é o tempo entre a queda e o  atendimento médico. De acordo com diversos estudos, a demora no atendimento está fortemente ligada ao índice de mortalidade e gravidade do acidente depois de uma queda. De acordo com X, quando ocorre o “long-lie”, ou seja, o idoso permanece mais de uma hora no chão a chance de que o idoso faleça antes dos 6 meses do ocorrido sobe para 50\%.

Visando minimizar essas graves consequências das quedas, diversos sistemas de detecção foram desenvolvidos nos últimos anos, porém estes sistemas não fazem uso de tecnologias mais popularizadas no mercado, ou utilizam de plataformas que não são vestíveis, prejudicando a mobilidade do usuário. 

Por exemplo, diversos sistemas utilizam o smartphone como principal plataforma na detecção de quedas. Analisando somente as questões de popularidade e hardware, o smartphone se apresenta uma solução plausível. De acordo com \cite{smartphoneSales:15}, foram vendidos  mais 1 bilhão de aparelhos somente em 2014. Na perspectiva de hardware, a maioria dos smartphones modernos possuem giroscópio ou  acelerômetro, dois dos principais sensores utilizados no reconhecimento de atividades atualmente.
 
Entretanto, quando pensamos em mobilidade, o smartphone passa a ser uma solução com baixo potencial,  pois para que os sistemas funcione corretamente o mesmo precisa está fixo em uma posição pré-estabelecida pelo sistema, como os bolso ou pulso do usuário \cite{FallDectionSmartPhone:12}. O que faz com que, em situações corriqueiras, como colocar o celular na bolsa, possa gerar um número grande de falsos positivos.

Este trabalho propõe como solução a criação de um sistemas de detecção de quedas através de uma solução integrada entre smartphone e o smartwatch. Na primeira vez que executar a aplicação, o usuário deverá cadastrar o nome e-mail dos usuários que ele deseja contactar em caso de uma  queda, feito este registro o usuário está pronto pra utilizar a aplicação. Ao detectar a queda, o smartwatch irá vibrar, e caso o usuário não indique que está bem, o sistema irá enviar um email com a localização do usuário para todos os contatos já cadastrados.   

O smartwatch é uma ferramenta que permite que este tipo de aplicação seja calma e invisível para o usuário, além de ter uma capacidade de processamento bastante similar aos smartphones com uma popularidade crescente. A Samsung, umas das empresas pioneiras no mercado de smartwatches, lançou em outubro de 2015 o Samsung Gear S2. O Gear é um exemplo de como esses sistemas estão cada vez mais poderosos. Ele possui uma memória RAM de 512 MB e 4GB de armazenamento,  conectividade WiFi e 4G além de diversos sensores como giroscópio e acelerômetro \cite{samsungSpecification:16}. A popularidade desta plataforma é vista através do número de smartwatches vendidos. No ano de 2015, 30,32 milhões de aparelhos foram vendidos, e a previsão é de que, em 2016, este número suba para 50,40 milhões. 


Os próximos capítulos estão organizados da seguinte maneira: O Capítulo \ref{cap:sistemasQuedas} apresenta os conceitos teóricos usados neste trabalho referente a Sistemas de Detecção de Quedas. O Capítulo \ref{cap:userModel} apresenta o SafeWatch, o sistema de Detecção de Quedas desenvolvido através de uma solução integrada entre o smartphone e o smartwatch. O Capítulo \ref{cap:avaliacao} apresenta o experimento realizado, e realiza a avaliação da ferramenta. Por fim, no capítulo \ref{cap:Conclusão}, seguem as conclusões e considerações finais. 

