\chapter{Conclusão}
\label{cap:conclusão}

A proposta deste trabalho foi criar um sistema de detecção de quedas que exige-se o mínimo de iteração possível do usuário, consumindo o mínimo de recursos possíveis com uma precisão similar aos demais \ac{FDS} existentes através de uma plataforma vestível que esteja se popularizando no mercado. 

As maiores dificuldades encontradas foram no desenvolvimento do algoritmo de detecção de quedas. Com a proposta inicial de somente o acelerômetro, diferente de demais sistemas que também utilizam o giroscópio a acurácia do sistema poderia ser afetada, caso algum dos limiares não se adaptassem a essa nova proposta. Outra dificuldade foi entender as caracteristicas de uma queda a partir de sua aceleração. Alguns conceitos físicos não são tão triviais, o que levou a muita pesquisa.  

Por fim, foi desenvolvido o SafeWatch, um sistema de detecção embarcado em um smartwatch que utiliza o smartphone como uma plataforma auxiliar. A interface é simples, e permite que o usuário interaja com o sistema de maneira fácil e somente quando necessário. O SafeWatch foi analisado em questão de perfomance e apresentou resultados satisfatórios com uma acurácia similar ou melhor que outros \ac{FDS} existentes.

Como trabalhos futuros, existem as seguintes possibilidades:

	\begin{enumerate}
		\item Buscar meios de otimizar o consumo de bateria do smartwatch mesmo com o constante monitoramento dos dados dos sensores.
		\item Integração com outros tipos de sensores, como o sensor de batimento cardiaco, afim de aumentar ainda mais a precisão do sistema.
		\item Reconhecimento de outros tipos de atividades além da queda (e.g detector de possível AVC),  visando expandir o sistema de um simples sistema de detector de quedas, para um sistema completo de monitoramento. 
		\item Realizar o teste da aplicação com um número maior de \ac{ADL}, principalmente aquelas que exigem uma movimentação maior do braço do individuo, como andar ou correr. 
 
	\end{enumerate}



