\chapter{Conclusão}
\label{cap:conclusão}

A proposta deste trabalho foi criar um sistema de detecção de quedas que exige-se o mínimo de interação possível do usuário, consumindo o mínimo de recursos possíveis com uma precisão similar aos demais \ac{SDQ} existentes através de uma plataforma vestível que esteja se popularizando no mercado. 

As maiores dificuldades encontradas foram no desenvolvimento do algoritmo de detecção de quedas. Com a proposta inicial de utilizar somente o acelerômetro, diferente de demais sistemas que também utilizam o giroscópio, a acurácia do sistema poderia ser afetada, caso algum dos limiares não se adaptassem a essa nova proposta. Outra dificuldade foi entender as características de uma queda a partir de sua aceleração, alguns conceitos físicos não são tão triviais, o que levou a muita pesquisa.    

Por fim, foi desenvolvido o SafeWatch, um sistema de detecção embarcado em um smartwatch que utiliza o smartphone como uma plataforma auxiliar. A interface é simples, e permite que o usuário interaja com o sistema de maneira fácil e somente quando necessário.

Utilizando somente um único sensor e um único dispositivo smartwatch, o SafeWatch apresentou resultados satisfatórios com uma acurácia similar ou melhor que outros \ac{SDQ} existentes.


Como trabalhos futuros, existem as seguintes possibilidades:

	\begin{enumerate}
		\item Buscar meios de otimizar o consumo de bateria do smartwatch mesmo com o constante monitoramento dos dados dos sensores.
		\item Integração com outros tipos de sensores, como o sensor de batimento cardíaco, a fim de aumentar ainda mais a acurácia do sistema.
		\item Reconhecimento de outros tipos de atividades além da queda (e.g detector de possível AVC),  visando expandir o sistema de um simples sistema de detector de quedas, para um sistema completo de monitoramento. 
		\item Utilizar outros meios de comunicação de emergência além do email. 
		\item Realizar o teste da aplicação com um número maior de \ac{AD}, principalmente aquelas que exigem uma movimentação maior do braço do individuo, como andar ou correr. 
		\item Realização de testes investigando o impacto de mais tipos de quedas na acurácia do sistema. Por exemplo, diferenciar quedas de um usuário consciente de um usuário inconsciente.
		\item Realizacão de experimentos com idosos, visando identificar particularidades em um evento de queda envolvendo está faixa etária. 
 
	\end{enumerate}



