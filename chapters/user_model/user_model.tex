\chapter{Modelo de Usuário} 
\label{cap:userModel}

O conjunto finito de propriedades e características do candidato a receber recomendações é o modelo do usuário. O modelo de usuário representa o usuário, ou seja, codifica suas preferências e necessidades \citep{ricci2011recommender}. Várias abordagens para modelagem de usuário têm sido utilizadas e, de certo modo, um sistema de recomendação pode ser visto como uma ferramenta que gera recomendações construindo e explorando modelos de usuário \citep{Berkovsky:2009:CMU:1499116.1499121}. Desde que não é possível fazer recomendações sem um modelo de usuário, a não ser que a recomendação não seja personalizada, o modelo de usuário sempre terá um papel central.

A informação que constitui o modelo de usuário pode variar de acordo com o domínio do sistema de recomendação. Por exemplo, no caso de um sistema de recomendação de filmes, o modelo será constituído pelos filmes que ele gosta de assistir, filmes que não gosta, os atores dos filmes que assiste, gênero etc.


\section{Feedback do Usuário}

Seja qual for a tecnologia específica explorada pelo sistema de recomendação, ela pode fornecer recomendações de alta qualidade apenas depois de ter modelado suas preferências \citep{Berkovsky:2008:MUM:1380736.1380749}. A tarefa de coletar dados de modelagem de usuário é normalmente realizada de duas maneiras:

\begin{itemize}
	\item{\textbf{Explícita}: Através do fornecimento das informações solicitadas explicitamente pelo usuário.}
	
	\item{\textbf{Implícita}: Através da aplicação de vários mecanismos de raciocínio que infere as informações necessárias com base no comportamento observável do usuário \citep{Hanani:2001:IFO:598287.598363}.}
\end{itemize}

A coleta explícita de dados de modelagem do usuário é considerada uma tarefa  precisa, mas que consome tempo e esforço, e por isso é normalmente evitada pelo usuário. Alternativamente, a coleta implícita envolve mecanismos de raciocínio automatizados, os quais podem interpretar errado o comportamento do usuário \citep{Berkovsky:2008:MUM:1380736.1380749}. Na prática, as abordagens explícitas e implícitas podem também ser combinadas.

De modo geral, a qualidade das recomendações fornecidas pelo usuário depende muito das características do modelo de usuário, por exemplo, quão preciso ele é, qual a quantidade de informação que ele armazena, e se essa informação está atualizada. Assim, como uma regra geral, quanto mais informação é armazenada no Modelo de Usuário, isto é, quanto mais conhecimento tem obtido sobre o usuário, melhor será a qualidade da recomendação. Neste contexto, qualidade refere-se à capacidade do sistema de sugerir exatamente aqueles produtos ou serviços que o usuário irá selecionar e comprar, ou predizer corretamente aqueles itens que o usuário irá gostar \citep{Berkovsky:2008:MUM:1380736.1380749}. Na prática, obter dados de modelagem de usuário suficientes para entregar recomendações de alta qualidade é difícil. Isso é especialmente importante nos estágios iniciais de interação com o usuário, quando pouca informação sobre o usuário está disponível. Nestes estágios, todas as técnicas de recomendação enfrentam o problema da "partida a frio", isto é, a situação onde a informação disponível sobre o usuário e/ou itens não é suficiente para fornecer recomendações de alta qualidade \citep{Linden:2003:ARI:642462.642471}.


\section{Modelagem}

Cada sistema de recomendação constrói e mantém uma coleção proprietária de Modelos de Usuário \citep{Montaner:2003:TRA:640471.640491}. Praticamente, isto significa que os dados de modelagem de usuário coletados são adaptados para:

\begin{itemize}
	\item{o conteúdo específico (produtos ou categoria de produtos) oferecido pelo sistema de recomendação, por exemplo, filmes, músicas, notícias, etc.}
	
	\item{a técnica de recomendação sendo explorada pelo sistema, por exemplo, filtragem colaborativa, filtragem baseada em conteúdo, ou alguma híbrida.}
\end{itemize}

Assim, uma enorme quantidade de dados heterogêneos de modelagem de usuário são disseminados entre vários sistemas. Entretanto, sistemas de recomendações na prática (especialmente os comerciais) não permitem outro sistema de recomendação externo acessá-los, como também não compartilha seus dados de modelagem de usuário. Apesar disso, é razoável supor que sistemas de recomendação podem potencialmente beneficiar-se de enriquecer seus dados de modelagem de usuário importando e integrando dados de modelagem de usuário coletados por outros sistemas de recomendação, e portanto fornecer melhores recomendações para o usuário \citep{Berkovsky:2008:MUM:1380736.1380749}.

\cite{Kobsa1994}, para evitar caracterizar modelagem de usuário através de estruturas e processos, listou os seguintes serviços frequentemente encontrados nos sistemas de modelagem de usuário:

\begin{itemize}
	\item{a representação de suposições sobre um ou mais tipos de características de usuário em modelo de usuários individuais (por exemplo, suposições sobre seu conhecimento, equívocos, objetivos, planos, preferências, tarefas, e habilidades);}
	
	\item{a representação de características relevantes comuns de usuários que pertencem a um específico subgrupo do sistema (o chamado, estereótipo);}
	
	\item{a classificação dos usuários como pertencentes a um ou mais destes subgrupos, e a integração das características típicas destes subgrupos dentro do modelo de usuário individual atual;}
	
	\item{o registro do comportamento dos usuários, particularmente sua interação passada com o sistema;}
	
	\item{a formação de suposições sobre o usuário baseado no histórico de interações;}
	
	\item{a generalização do histórico de interações de muitos usuários em estereótipos;}
	
	\item{o desenho de suposições adicionais sobre o usuário atual baseado em valores iniciais.}
	
	\item{manutenção de consistência no modelo de usuário;}
	
	\item{o fornecimento da suposição atual sobre o usuário, como também justificativas para estas suposições;}
	
	\item{a avaliação das entradas no modelo de usuário atual, e a comparação com padrões.}
\end{itemize}

Segundo \cite{Kobsa:2001:GUM:598284.598347}, os seguintes requisitos para os sistemas de modelagem de usuário são considerados importantes:

\begin{itemize}
	\item{\textbf{Generalidade, incluindo independência de domínio}: Os sistemas precisam ser utilizáveis por quantas aplicações e domínios de conteúdo for possível, e dentro destes domínios por quantas tarefas de modelagem de usuário for possível.}
	
	\item{\textbf{Expressividade}: Os sistemas precisam expressar quantos tipos de suposições sobre o usuário forem possíveis ao mesmo tempo.}
	
	\item{\textbf{Capacidade Inferencial Forte}: Os sistemas precisam realizar todo tipo de raciocínio que são tradicionalmente distinguidos em inteligência artificial e lógica formal.}
\end{itemize}






\section{Modelos de Usuário Baseados em Múltiplos Domínios}

Sistemas de recomendação é um campo de pesquisa ativo e vem sendo utilizado com sucesso em um grande número de sistemas como Netflix\footnote{http://www.netflix.com}, Youtube\footnote{http://www.youtube.com}, iTunes\footnote{http://www.apple.com/itunes/}. A grande maioria destes sistemas oferecem recomendações apenas para itens pertencentes a um único domínio. Nestes casos, as recomendações são computadas utilizando feedback do usuário (avaliação) sobre itens no domínio alvo. Em sites de \textit{e-commerce}, como Amazon\footnote{http://www.amazon.com}, no entanto, seria útil explorar avaliações do usuário sobre diversos tipos de itens para gerar um modelo mais geral das preferências do usuário. De fato, pode existir dependências e correlações entre as preferências entre diferentes domínios e ao invés de tratar cada tipo (por exemplo, eletrônicos e música) independentemente, o conhecimento do usuário adquirido em um domínio pode ser transferido e explorado em muitos outros domínios \citep{fernandez2012cross}.

Além disso, um sistema pode oferecer recomendações personalizadas de itens em múltiplos domínios, por exemplo, sugerir não apenas um filme em particular, mas também CDs de música, livros ou videogames que de alguma maneira estejam relacionados com o filme. Analogamente, em uma aplicação turística seria de grande valor sugerir um evento cultural para cliente que tenha reservado um quarto em um hotel recomendado, ou em sistema de e-learning, apresentar um estudante com referências bibliográficas relacionadas a uma vídeo-aula que tenha sido recentemente recomendada \citep{fernandez2012cross}.

Um dos primeiros estudos relacionados a recomendações em múltiplos domínios foi apresentado por \citep{Winoto2008}. Eles identificaram três importantes questões a serem investigadas:

\begin{itemize}
	\item{Verificar a existência de correlações globais das preferências do usuário para itens em diferentes domínios.}
	
	\item{Criar modelos capazes de explorar as preferências de usuário em um domínio fonte para predizer as preferências de usuário em domínio alvo}
	
	\item{Desenvolver avaliações apropriadas para recomendações em múltiplos domínios.}
\end{itemize}

\cite{Winoto2008} acreditavam que, embora recomendações entre múltiplos domínios tendessem a ser menos precisas do que recomendações em um único domínio, a primeira seria mais diversificada, o que pode levar a uma maior satisfação do usuário \citep{Adomavicius:2005:TNG:1070611.1070751}. Além disso, recomendações entre múltiplos domínios tem outras vantagens, como lidar com o problema da partida-a-frio (cold-start problem) \citep{Abel2012}, e também o problema da esparsidade \citep{Li:2009:MBC:1661445.1661773}, \citep{AAAI101649}. Ao identificar as relações entre itens em dois domínios diferentes, pode-se sugerir para um usuário com itens em um domínio inexplorado, simplesmente explorando suas preferências para os itens em outros domínios conhecidos.





\section{Ranking Personalizado}
O objetivo de um ranking personalizado é fornecer para um usuário uma lista ordenada de itens. Um exemplo é uma loja online que pretende recomendar uma lista personalizada e ranqueada de itens que o usuário poderia querer comprar. Em sistemas de feedback implícito, apenas observações positivas estão disponíveis.  Os itens não avaliados pelo usuário (por exemplo, um usuário que não tenha comprado um item ainda) são uma mistura de feedback negativo real (o usuário não está interessado em comprar o item) e falta de valores (o usuário pode querer comprar o item no futuro).


\subsection{Bayesian Personalized Ranking}

\ac{BPR} é um framework genérico para otimizar diferentes tipos de modelos baseado em treinar dados contendo apenas feedbacks implícitos. \ac{BPR} é baseado na ideia de reduzir a classificação para pares de classificação \citep{balcan2008}. Foi proposto por \citep{Rendle:2009:BBP:1795114.1795167} para lidar com o problema ao treinar um modelo de recomendação de itens utilizando apenas feedbacks implícitos baseado apenas em dados positivos e negativos. O modelo será levado a fornecer pontuações positivas para os itens observados, enquanto considerará itens não visitados como negativo. Entretanto, tal suposição é imprecisa porque um item não observado pode ser pelo fato de que não é conhecido pelo usuário.

Considerando este problema, ao invés de treinar o modelo utilizando apenas os pares usuário-item, \cite{Rendle:2009:BBP:1795114.1795167} propôs considerar a ordem relativa entre um par de itens, de acordo com as preferências do usuário. Sendo $N(u)$ o conjunto de itens para o qual o usuário $u$ forneceu feedback implícito e $\bar{N}(u)$ o conjunto de itens desconhecidos pelo usuário $u$, é inferido que se um item $i$ foi visto por um usuário $u$ e $j$ não foi visto ($i \in N(u)~$ e $j \in \bar{N}(u)$), então $i >_{u} j$, o que significa que ele prefere $i$ ao invés de $j$.


\subsection{BPR-Linear}
\label{sec:bpr-linear}
O BPR-Linear \citep{gantner2010} é um algoritmo baseado no framework \ac{BPR}, que utiliza atributos de item em um mapeamento linear para estimar pontuação. A regra de predição é definida como:

\begin{equation}
\hat{r}_{ui} = \phi_f(\vec{a}_i) = \displaystyle\sum_{g=1}^{n} w_{ug} a_{ig}~~,
\label{eq:bpr-linear}
\end{equation}

\noindent onde $\phi_f : \mathbb{R}^n \rightarrow \mathbb{R}$ é uma função que mapeia os atributos do item para as preferências gerais $\hat{r}_{ui}$ e $\vec{a}_i$ é um vetor booleano de tamanho $n$ onde cada elemento $a_{ig}$ representa a ocorrência ou não de um atributo, e $w_{ug}$ é uma matriz de peso gerado utilizando LearnBPR, que é uma variação da técnica "stochastic gradient descent" \cite{gantner2011}. Desta maneira, nós primeiro calculamos a importância relativa entre dois itens:

\begin{equation}
\begin{array}{ll}
\hat{s}_{uij} &= \hat{r}_{ui} - \hat{r}_{uj} \\
&= \displaystyle\sum_{g=1}^{n} w_{ug} a_{ig} - \displaystyle\sum_{g=1}^{n} w_{ug} a_{jg} \\
&= \displaystyle\sum_{g=1}^{n} w_{ug} (a_{ig} - a_{jg})~~.
\end{array}
\end{equation}

Finalmente, a derivada parcial de $w_{ug}$ é feita:

\begin{equation}
\frac{\partial}{\partial w_{ug}}\hat{s}_{uij} = (a_{ig} - a_{jg})~~,
\end{equation}

\noindent que é aplicada para o Algoritmo LearnBPR considerando que $\Theta = (w_{*})$ para todo o conjunto de usuário e descrições.




\subsection{BPR-Mapping}
\label{sec:brp-mapping}
O BPR-Mapping é um algoritmo proposto por \citep{gantner2010}, baseado no framework \ac{BPR}, para personalizar um ranking de itens utilizando apenas feedback implícito. A principal diferença é que ele utiliza o mapeamento linear descrito em \ref{sec:bpr-linear} para otimizar os fatores de item que serão utilizados depois em uma regra de predição de fatoração de matriz extendida. Essa fatoração de matriz extendida foi otimizada por Bayesian Personalized Ranking (BPR-MF) \cite{Rendle:2009:BBP:1795114.1795167} que pode lidar com o problema da "partida a frio"~ (cold-start problem), produzindo de maneira rápida e precisa atributo contextual de item de recomendação.


\cite{gantner2010} objetiva o caso onde novos usuários e itens são adicionados primeiramente calculando os vetores de característica latentes de atributos como a idade do usuário ou o gênero do filme, e então utilizando os vetores de característica latentes estimados para calcular a pontuação do modelo da matriz de fatorização (MF) subjacente.

O modelo considera a regra de predição de fatoração de matriz:

\begin{equation}
\hat{r}_{ui} = b_{ui} + p^T_u q_i = b_{ui} + \displaystyle\sum_{f=1}^{k}p_{uf} q_{if}~~,
\end{equation}

\noindent onde cada usuário $u$ é associado a um vetor usuário-fatores $p_u \in \mathbb{R}^f$, e cada item $i$ com um vetor $q_i \in \mathbb{R}^f$. A base $b_{ui}$ é definida como $b_{ui} = \mu + b_u + b_i$ e indica as estimativas distintas de usuário e itens em comparação com a média de classificação global $\mu$.

Deste modelo, os fatores de item são mapeados de acordo com seus atributos como:

\begin{equation}
\hat{r}_{ui} = b_{ui} + \displaystyle\sum_{f=1}^{k}p_{uf} \phi_f(\vec{a}_i)~~,
\end{equation}

\noindent onde $\phi_f(\vec{a}_i)$ tem a mesma definição que a Equação \ref{eq:bpr-linear}. 



\section{Trabalhos Relacionados}

Em um sistema de recomendação, como um de venda de livros, pode não haver registros suficientes de classificações de usuário-item para um novo usuário ou um novo produto, isto é chamado de esparsidade de dados, onde a informação útil está dispersa e pouca. Uma solução seria pedir ao usuário que classificasse alguns itens, porém, isto poderia prejudicar a experiência do usuário. Pesquisas tem sido feitas para descobrir métodos que permitam integrar informações de redes sociais ao processo de recomendação.

Normalmente, os Modelos de Usuário são específicos para o conteúdo oferecido por um serviço. Em 2005, \cite{Berkovsky:2005:EPM:2153634.2153658} propôs um mecanismo de mediação de Modelos de Usuário entre diferentes domínios de entretenimento, com o objetivo de enriquecer o Modelo de Usuário de um serviço através da importação e integração de Modelos de Usuários parciais construídos por outros serviços, o que foi chamado de \textit{Cross-Domain User Modeling}. Um \textit{mediador} cria um Modelo de Usuário de acordo com a necessidade do serviço alvo, onde deveria determinar uma representação do modelo, identificar os serviços que possam fornecer os dados necessários e integrar os Modelos de Usuário parciais e criar um Modelo de Usuário para o serviço alvo.

Em 2009, pesquisadores propuseram um método para resolver o problema da esparsidade de dados \citep{Li:2009:MBC:1661445.1661773}. Estes métodos tinham por objetivo utilizar dados de outros sistemas de recomendação, chamados de domínio auxiliar, e transferir seus conhecimentos para um domínio alvo. Os padrões de classificação de usuário-item eram aprendidos através de uma matriz de classificação auxiliar densa e transferidas para uma matriz de classificação alvo esparsa. Essa coleção de padrões foi chamada de "codebook".

Um estudo realizado por \cite{Wang2010}, utilizou conteúdo de redes sociais para a recomendação de itens. Neste trabalho, eram recomendados usuários e atividades para membros de redes sociais, misturando informação de várias redes sociais em que o usuário estava registrado. Diferentemente do nosso trabalho, o sistema deles envolvia feedback explícito do usuário, onde era necessário fornecer informações (gostou/não gostou) sobre as atividades recomendadas. No trabalho realizado aqui, todo os dados foram coletados de forma automática.

O trabalho realizado por \cite{Ma:2011:RSS:1935826.1935877}, extrai dados de preferências do usuário e também suas amizades em redes sociais. No que diz respeito a extração dessas preferências, o trabalho é semelhante ao nosso, porém eles extraem classificações explícitas de sites de \textit{reviews} de usuários. Em nosso trabalho, o interesse do usuário é inferido pelos filmes que ele marcou no Facebook\footnote{http://www.facebook.com} ter assistido, e livros que marcou ter lido. Outra diferença é que olhamos também as preferências do usuário em outros domínios, para realizar a recomendação em múltiplos domínios.

O framework proposto por \cite{tobias2013semantic} tinha por objetivo construir redes semânticas automaticamente, que conectam itens de diferentes domínios através de relações explícitas. Essas redes forneceriam recomendações de itens em um domínio alvo considerando as preferências do usuário em um outro domínio distinto. O funcionamento consistia de 4 estágios: Representação do Conhecimento, Extração do Conhecimento, Parentesco Semântico e Recomendação. Contrário ao nosso trabalho, o usuário tinha que informar suas preferências (por gêneros musicais) em um formulário disponível em uma aplicação Web.

Neste trabalho, nós implementamos um sistema que coleta informações de múltiplos domínios sobre usuários a partir do Facebook\footnote{http://www.facebook.com} de forma automática, para a criação de um Modelo de Usuário baseado em suas preferências por filmes e livros.



\section{Sumário}

Neste capítulo, fizemos uma visão geral sobre Modelos de Usuário. Inicialmente tratamos sobre feedback do usuário, onde vimos como coletar dados do usuário. Em seguida discutimos sobre modelagem do usuário e suas especificidades. Abordamos aqui também, os modelos de usuário baseados em múltiplos domínios. Falamos sobre ranking personalizado, onde vimos os algoritmos BPR-Linear e BPR-Mapping baseados no framework Bayesian Personalized Ranking. Por fim, discutimos trabalhos relacionados. No Capítulo \ref{cap:multiDomainUserModel} falaremos sobre a proposta deste trabalho, Modelo de Usuário em Múltiplos Domínios Automatizado. Será introduzido do que se trata, arquitetura, tecnologias utilizadas, implementação.