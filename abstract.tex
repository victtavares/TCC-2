
Falling can have serious consequences, enough to be considered a serious public health problem that affects mainly the older population, in which is related to the loss of confidence, self-esteem and autonomy.This problem is shown even more relevant if we consider the growing number of seniors, who in the search of their independence and autonomy decide to live alone. It is crucial that the elderly have quick access to medical care, a key part for  quick recovery. The delay in medical care is linked with the increase of mortality rates and severity in a fall event. Thinking about it, the \textit{SafeWatch} was developed as a fall detection system embedded in smartwatches. The proposed system will monitor the seniors through sensors at the smartwatch, and when a fall is detected, It will vibrate on the user's wrist and report to a list of emergency contacts their location and inform the possibility of the elderly to be in a dangerous situation. Experiments were performed with eight individuals of different biotypes, in which each one of them simulated a fall event in different directions. According to the experiments, it was possible to evaluate the application reliability through the values of \textit{Sensitivity} and \textit{Specificity} that reached 89,06\% and 100\%, respectively.



\begin{keywords}
	Smartwatch, Ubiquitous Computing,  Fall, Elderly
\end{keywords}